\documentclass[English,Chinese,French,puretext]{einfart}

\CreateTheorem{definition}<highest>
\CreateTheorem{theorem}[definition]

\linenumbers % Enable line numbers

%%================================
%% Import toolkit
%%================================
\usepackage{ProjLib}
\usepackage{longtable}  % breakable tables
\usepackage{hologo}     % more TeX logo
\usetikzlibrary{calc}

\usepackage{blindtext}

\UseLanguage{French}

%%================================
%% For typesetting code
%%================================
\usepackage{listings}
\definecolor{maintheme}{RGB}{70,130,180}
\definecolor{forestgreen}{RGB}{21,122,81}
\definecolor{lightergray}{gray}{0.99}
\lstset{language=[LaTeX]TeX,
    keywordstyle=\color{maintheme},
    basicstyle=\ttfamily,
    commentstyle=\color{forestgreen}\ttfamily,
    stringstyle=\rmfamily,
    showstringspaces=false,
    breaklines=true,
    frame=lines,
    backgroundcolor=\color{lightergray},
    flexiblecolumns=true,
    escapeinside={(*}{*)},
    % numbers=left,
    numberstyle=\scriptsize, stepnumber=1, numbersep=5pt,
    % firstnumber=last,
}
\providecommand{\meta}[1]{$\langle${\normalfont\itshape#1}$\rangle$}
\lstset{moretexcs=%
    {linenumbers,nolinenumbers,part,parttext,chapter,section,subsection,subsubsection,frontmatter,mainmatter,backmatter,tableofcontents,href,
    color,NameTheorem,CreateTheorem,proofideanameEN,cref,dnf,needgraph,UseLanguage,UseOtherLanguage,AddLanguageSetting,maketitle,address,curraddr,email,keywords,subjclass,thanks,dedicatory,PLdate,ProjLib,qedhere
    }
}
\lstnewenvironment{code}%
{\setstretch{1.07}\LocallyStopLineNumbers%
\setkeys{lst}{columns=fullflexible,keepspaces=true}%
}
{\ResumeLineNumbers}
\lstnewenvironment{code*}%
{\setstretch{1.07}\LocallyStopLineNumbers%
\setkeys{lst}{numbers=left,columns=fullflexible,keepspaces=true}%
}
{\ResumeLineNumbers}

%%================================
%% tip
%%================================
\usepackage[many]{tcolorbox}
\newenvironment{tip}[1][Astuce]{%
    \LocallyStopLineNumbers%
    \begin{tcolorbox}[breakable,
        enhanced,
        width = \textwidth,
        colback = paper, colbacktitle = paper,
        colframe = gray!50, boxrule=0.2mm,
        coltitle = black,
        fonttitle = \sffamily,
        attach boxed title to top left = {yshift=-\tcboxedtitleheight/2, xshift=.5cm},
        boxed title style = {boxrule=0pt, colframe=paper},
        before skip = 0.3cm,
        after skip = 0.3cm,
        top = 3mm,
        bottom = 3mm,
        title={\scshape\sffamily #1}]%
}{\end{tcolorbox}\ResumeLineNumbers}

%%================================
%% Names
%%================================
\providecommand{\minimalist}{\textsf{minimalist}}
\providecommand{\minimart}{\textsf{minimart}}
\providecommand{\minimbook}{\textsf{minimbook}}
\providecommand{\einfart}{\textsf{einfart}}
\providecommand{\simplivre}{\textsf{simplivre}}

%%================================
%% Titles
%%================================
\let\LevelOneTitle\section
\let\LevelTwoTitle\subsection
\let\LevelThreeTitle\subsubsection

%%================================
%% Main text
%%================================
\begin{document}

\title{\einfart{}, écrivez vos articles de manière simple et claire}
\author{Jinwen XU}
\thanks{Correspondant à : \texttt{\einfart{} 2021/07/13}}
\email{\href{mailto:ProjLib@outlook.com}{ProjLib@outlook.com}}
\date{juillet 2021, à Pékin}

\maketitle

\begin{abstract}
    \einfart{} fait partie de la série de classes \minimalist{}, dont le nom est tiré du mot allemand « einfach » (simple), combiné avec les trois premières lettres de « artikel » (article) . L'ensemble de la collection comprend \minimart{} et \einfart{} pour la composition d'articles, et \minimbook{} et \simplivre{} pour celle des livres. Mon intention initiale en les concevant était d'écrire des brouillons et des notes qui semblent simples mais pas grossières.

    \einfart{} prend en charge plusieurs langues, notamment le chinois (simplifié et traditionnel), l'anglais, le français, l'allemand, l'italien, le japonais, le portugais (européen et brésilien), le russe et l'espagnol. Ces langues peuvent être commutées de manière transparente dans un seul document. En raison de l'utilisation de polices personnalisées, \einfart{} demande soit \hologo{XeLaTeX} soit \hologo{LuaLaTeX} pour la compilation.

    Cette documentation est composée à l'aide de \einfart{}. Vous pouvez le considérer comme une courte introduction et une démonstration.
\end{abstract}


\setcounter{tocdepth}{2}
\tableofcontents


\LevelOneTitle*{Avant de commencer}

Pour utiliser les classes de documents décrites ici, vous devez :
\begin{itemize}
      \item installer TeX Live ou MikTeX de la dernière version possible, et vous assurer que \texttt{minimalist} et \texttt{projlib} sont correctement installés dans votre système \TeX{}.
      \item être familiarisé avec l'utilisation de base de \LaTeX{}, et savoir comment compiler vos documents avec \hologo{pdfLaTeX}, \hologo{XeLaTeX} ou \hologo{LuaLaTeX}.
\end{itemize}


\LevelOneTitle{Utilisation et exemples}

\LevelTwoTitle{Comment l'ajouter}

Il suffit simplement de mettre

\begin{code}
  \documentclass{einfart}
\end{code}

comme première ligne pour utiliser la classe \einfart{}. Veuillez noter que vous devez utiliser le moteur \hologo{XeLaTeX} ou \hologo{LuaLaTeX} pour compiler.

\LevelTwoTitle{Exemple - Un document complet}

Regardons d'abord un document complet.

\begin{code*}
\documentclass{einfart}
\usepackage{ProjLib}

\UseLanguage{French}

\begin{document}

\title{(*\meta{title}*)}
\author{(*\meta{author}*)}
\date{\PLdate{2022-04-01}}

\maketitle

\begin{abstract}
    Ceci est un résumé. \dnf<Plus de contenu est nécessaire.>
\end{abstract}
\begin{keyword}
    AAA, BBB, CCC, DDD, EEE
\end{keyword}

\section{Un théorème}

\begin{theorem}\label{thm:abc}
    Ceci est un théorème.
\end{theorem}
Référence du théorème: \cref{thm:abc}

\end{document}
\end{code*}


Si vous trouvez cela un peu compliqué, ne vous inquiétez pas. Examinons maintenant cet exemple pièce par pièce.

\clearpage
\LevelThreeTitle{Initialisation}

\medskip
\begin{code}
\documentclass{einfart}
\usepackage{ProjLib}
\end{code}

L'initialisation est simple. La première ligne ajoute la classe de document \einfart{}, et la deuxième ligne ajoute la boîte à outils \ProjLib{} pour obtenir des fonctionnalités supplémentaires.

\LevelThreeTitle{Choisir la langue}

\medskip
\begin{code}
\UseLanguage{French}
\end{code}

Cette ligne indique que le français sera utilisé dans le document (d'ailleurs, si seul l'anglais apparaît dans votre article, alors il n'est pas nécessaire de choisir la langue). Vous pouvez également changer de langue de la même manière plus tard au milieu du texte. Les langues prises en charge sont les suivantes : chinois simplifié, chinois traditionnel, japonais, anglais, français, allemand, espagnol, portugais, portugais brésilien et russe.

Pour une description détaillée de cette commande et d'autres commandes associées, veuillez vous référer à la section sur le support multilingue.

\LevelThreeTitle{Titre, informations sur l'auteur, résumé et mots-clés}

\medskip
\begin{code}
\title{(*\meta{title}*)}
\author{(*\meta{author}*)}
\date{\PLdate{2022-04-01}}
\maketitle

\begin{abstract}
    (*\meta{abstract}*)
\end{abstract}
\begin{keyword}
    (*\meta{keywords}*)
\end{keyword}
\end{code}

Cette partie commence par le titre et le bloc d'informations sur l'auteur. L'exemple montre l'utilisation de base, mais en fait, vous pouvez également écrire comme :

\begin{code}
\author{(*\meta{author 1}*)}
\address{(*\meta{address 1}*)}
\email{(*\meta{email 1}*)}
\author{(*\meta{author 2}*)}
\address{(*\meta{address 2}*)}
\email{(*\meta{email 2}*)}
...
\end{code}

De plus, vous pouvez également écrire à la manière \AmS{}, c'est-à-dire :

\begin{code}
\title{(*\meta{title}*)}
\author{(*\meta{author 1}*)}
\address{(*\meta{address 1}*)}
\email{(*\meta{email 1}*)}
\author{(*\meta{author 2}*)}
\address{(*\meta{address 2}*)}
\email{(*\meta{email 2}*)}
\date{\PLdate{2022-04-01}}
\subjclass{*****}
\keywords{(*\meta{keywords}*)}

\begin{abstract}
    (*\meta{abstract}*)
\end{abstract}

\maketitle
\end{code}

\LevelThreeTitle{Marques de brouillon}

\medskip
\begin{code}
\dnf<(*\meta{some hint}*)>
\end{code}
Lorsque vous avez des endroits qui ne sont pas encore finis, vous pouvez les marquer avec cette commande, ce qui est particulièrement utile lors de la phase de brouillon.

\LevelThreeTitle{Environnements de type théorème}

\medskip
\begin{code}
\begin{theorem}\label{thm:abc}
    Ceci est un théorème.
\end{theorem}
Référence du théorème: \cref{thm:abc}
\end{code}

Les environnements de type théorème couramment utilisés ont été prédéfinis. De plus, lors du référencement d'un environnement de type théorème, il est recommandé d'utiliser \lstinline|\cref{|\meta{label}\texttt{\}} --- de cette manière, il ne serait pas nécessaire d'écrire explicitement le nom de l'environnement correspondant à chaque fois.

\begin{tip}
Si vous souhaitez utiliser la classe standard à la place plus tard, remplacez simplement les deux premières lignes par :

\begin{code}
\documentclass{article}
\usepackage[a4paper,margin=1in]{geometry}
\usepackage[hidelinks]{hyperref}
\usepackage[palatino,amsfashion]{ProjLib}
\end{code}

ou utilisez la classe \AmS{} :

\begin{code}
\documentclass{amsart}
\usepackage[a4paper,margin=1in]{geometry}
\usepackage[hidelinks]{hyperref}
\usepackage[palatino]{ProjLib}
\end{code}
\vspace{-.5\baselineskip}
\end{tip}

\begin{tip}
Si vous aimez la classe de document actuelle, mais que vous souhaitez un style plus « simple », vous pouvez utiliser l'option \texttt{classical}, comme ceci :

\begin{code}
\documentclass[classical]{einfart}
\end{code}
\vspace{-.5\baselineskip}
\end{tip}


\LevelOneTitle{À propos des polices par défaut}
Par défaut, cette classe de document utilise Palatino Linotype comme police anglaise, YouSong et YouHei GBK de FounderType comme polices chinoises\footnote{Pour plus de détails, veuillez visiter le site Web de FounderType : \url{https://www.foundertype.com}.}, et utilise partiellement Neo Euler comme police mathématique. Parmi eux, Neo Euler peut être téléchargé sur \url{https://github.com/khaledhosny/euler-otf}. Les autres polices ne sont pas gratuites, vous devez les acheter et les installer vous-même.


Lorsque la police correspondante n'est pas installée, les polices fournies avec TeX Live seront utilisées à la place. Dans ce cas, l'expérience peut être réduite.


\LevelOneTitle{Les options}


\begin{itemize}
    \item Les options de langue \texttt{EN} / \texttt{english} / \texttt{English}, \texttt{FR} / \texttt{french} / \texttt{French}, etc.
        \begin{itemize}
            \item Pour les noms d'options d'une langue spécifique, veuillez vous référer à \meta{language name} dans la section suivante. La première langue spécifiée sera considérée comme la langue par défaut.
            \item Les options de langue ne sont pas nécessaires, elles servent principalement à augmenter la vitesse de compilation. Sans eux, le résultat serait le même, justement plus lent.
        \end{itemize}
    \item \texttt{draft} ou \texttt{fast}
        \begin{itemize}
            \item L'option \verb|fast| permet un style plus rapide mais légèrement plus rugueux, les principales différences sont :
            \begin{itemize}
                \item Utilisez une configuration de police mathématique plus simple ;
                \item N'utilisez pas \textsf{hyperref} ;
                \item Activez le mode rapide de la boîte à outils \ProjLib{}.
            \end{itemize}
        \end{itemize}
    \begin{tip}
        Pendant la phase de brouillon, il est recommandé d'utiliser le \verb|fast| option pour accélérer la compilation. Quand dans \verb|fast| mode, il y aura un filigrane ``DRAFT'' pour indiquer que vous êtes actuellement en mode brouillon.
    \end{tip}
    \item \texttt{a4paper} ou \texttt{b5paper}
        \begin{itemize}
            \item Options de format de papier. Le format de papier par défaut est 7 pouces $\times$ 10 pouces.
        \end{itemize}
    \item \texttt{palatino}, \texttt{times}, \texttt{garamond}, \texttt{biolinum} ~$|$~ \texttt{useosf}
        \begin{itemize}
            \item Options de police. Comme son nom l'indique, la police avec le nom correspondant sera utilisée.
            \item L'option \texttt{useosf} est pour activer les chiffres à l'ancienne.
        \end{itemize}
    \item \texttt{allowbf}
        \begin{itemize}
            \item Pour activer les titres en gras. Lorsque cette option est utilisée, le titre principal, les titres de tous les niveaux et les noms des environnements de type théorème seront en gras.
        \end{itemize}
    \item \texttt{classical}
        \begin{itemize}
            \item Mode classique. Lorsque cette option est utilisée, le style deviendra plus régulier : les paragraphes sont en retrait, l'utilisation de soulignements est réduite, les styles de titres sont modifiés et les styles de théorème seront beaucoup plus proches des styles par défaut.
        \end{itemize}
    \begin{tip}
        \texttt{allowbf} + \texttt{classical} est probablement un bon choix si vous préférez le style traditionnel.
    \end{tip}
    \item \texttt{useindent}
        \begin{itemize}
            \item Utilisez l'indentation des paragraphes au lieu de l'espacement entre les paragraphes.
        \end{itemize}
    \item \texttt{runin}
        \begin{itemize}
            \item Utilisez le style « runin » pour \lstinline|\subsubsection|
        \end{itemize}
    \item \texttt{puretext} ou \texttt{nothms}
        \begin{itemize}
            \item Mode texte pur. Ne pas définir les environnements de type théorème.
        \end{itemize}
    \item \texttt{nothmnum}
        \begin{itemize}
            \item Ne pas numéroter les environnements de type théorème.
        \end{itemize}
\end{itemize}

\LevelOneTitle{Instructions par sujet}

\LevelTwoTitle{Configurer la langue}

\einfart{} prend en charge plusieurs langues, notamment le chinois (simplifié et traditionnel), l'anglais, le français, l'allemand, l'italien, le japonais, le portugais (européen et brésilien), le russe et l'espagnol. La langue peut être sélectionnée par les macros suivantes :

\begin{itemize}
    \item \lstinline|\UseLanguage{|\meta{language name}\lstinline|}| est utilisé pour spécifier la langue. Le réglage correspondant de la langue sera appliqué après celui-ci. Il peut être utilisé soit dans le préambule ou dans le texte. Lorsqu'aucune langue n'est spécifiée, « English » est sélectionné par défaut.
    \item \lstinline|\UseOtherLanguage{|\meta{language name}\lstinline|}{|\meta{content}\lstinline|}|, qui utilise les paramètres de langue spécifiés pour composer \meta{content}. Par rapport à \lstinline|\UseLanguage|, il ne modifiera pas l'interligne, donc l'interligne restera stable lorsque le texte CJK et occidental sont mélangés.
\end{itemize}

\clearpage
\meta{language name} peut être (il n'est pas sensible à la casse, par exemple, \texttt{French} et \texttt{french} ont le même effet) :
\begin{itemize}
    \item chinois simplifié : \texttt{CN}, \texttt{Chinese}, \texttt{SChinese} ou \texttt{SimplifiedChinese}
    \item chinois traditionnel : \texttt{TC}, \texttt{TChinese} ou \texttt{TraditionalChinese}
    \item anglais : \texttt{EN} ou \texttt{English}
    \item français : \texttt{FR} ou \texttt{French}
    \item allemand : \texttt{DE}, \texttt{German} ou \texttt{ngerman}
    \item italien : \texttt{IT} ou \texttt{Italian}
    \item portugais : \texttt{PT} ou \texttt{Portuguese}
    \item portugais (brésilien) : \texttt{BR} ou \texttt{Brazilian}
    \item espagnol : \texttt{ES} ou \texttt{Spanish}
    \item japonais : \texttt{JP} ou \texttt{Japanese}
    \item russe : \texttt{RU} ou \texttt{Russian}
\end{itemize}

\medskip
De plus, vous pouvez également ajouter de nouveaux paramètres à la langue sélectionnée :
\begin{itemize}
    \item \lstinline|\AddLanguageSetting{|\meta{settings}\lstinline|}|
    \begin{itemize}
        \item Ajoutez \meta{settings} à toutes les langues prises en charge.
    \end{itemize}
    \item \lstinline|\AddLanguageSetting(|\meta{language name}\lstinline|){|\meta{settings}\lstinline|}|
    \begin{itemize}
        \item Ajoutez \meta{settings} à la langue \meta{language name} sélectionnée.
    \end{itemize}
\end{itemize}
Par exemple, \lstinline|\AddLanguageSetting(German){\color{orange}}| peut rendre tout le texte allemand affiché en orange (bien sûr, il faut alors ajouter \lstinline|\AddLanguageSetting{\color{black}}| afin de corriger la couleur du texte dans d'autres langues).

\LevelTwoTitle{Théorèmes et comment les référencer}

Des environnements tels que \texttt{definition} et \texttt{theorem} ont été prédéfinis et peuvent être utilisés directement.

Plus précisement, les environnements prédéfinis incluent : \texttt{assumption}, \texttt{axiom}, \texttt{conjecture}, \texttt{convention}, \texttt{corollary}, \texttt{definition}, \texttt{definition-proposition}, \texttt{definition-theorem}, \texttt{example}, \texttt{exercise}, \texttt{fact}, \texttt{hypothesis}, \texttt{lemma}, \texttt{notation}, \texttt{observation}, \texttt{problem}, \texttt{property}, \texttt{proposition}, \texttt{question}, \texttt{remark}, \texttt{theorem}, et la version non numérotée correspondante avec un astérisque \lstinline|*| dans le nom. Les titres changeront avec la langue actuelle. Par exemple, \texttt{theorem} sera affiché comme « Theorem » en mode anglais et « Théorème » en mode français.

Lors du référencement d'un environnement de type théorème, il est recommandé d'utiliser \lstinline|\cref{|\meta{label}\texttt{\}}. De cette façon, il n'est pas nécessaire d'écrire explicitement le nom de l'environnement correspondant à chaque fois.

\begin{tip}[Exemple]
\begin{code}
  \begin{definition}[Des choses étranges] \label{def: strange} ...
\end{code}

will produce
\begin{definition}[Des choses étranges]\label{def: strange}
    C'est la définition de certains objets étranges. Il y a approximativement un espace d'une ligne avant et après l'environnement de type théorème, et il y aura un symbole pour marquer la fin de l'environnement.
\end{definition}

\lstinline|\cref{def: strange}| s'affichera sous la forme : \cref{def: strange}.

Après avoir utilisé \lstinline|\UseLanguage{French}|, un théorème s'affichera sous la forme :

\UseLanguage{English}
\begin{theorem}[Useless]\label{thm}
    A theorem in English.
\end{theorem}

Par défaut, lorsqu'il est référencé, le nom de l'environnement de type théorème correspond toujours à la langue du contexte dans lequel se trouve l'environnement. Par exemple, la définition ci-dessus est toujours affichée en français dans le mode anglais courant : \cref{def: strange} et \cref{thm}. Si vous voulez que le nom du théorème corresponde au contexte actuel lors du référencement, vous pouvez ajouter \texttt{regionalref} aux options globales.
\end{tip}

\UseLanguage{French}

\LevelTwoTitle{Définir un nouvel environnement de type théorème}

Si vous avez besoin de définir un nouvel environnement de type théorème, vous devez d'abord définir le nom de l'environnement dans le langage à utiliser :
\begin{itemize}
    \item \lstinline|\NameTheorem[|\meta{language name}\lstinline|]{|\meta{name of environment}\lstinline|}{|\meta{name string}\lstinline|}|
\end{itemize}
Pour \meta{language name}, veuillez vous référer à la section sur la configuration de la langue. Lorsqu'il n'est pas spécifié, le nom sera défini pour toutes les langues prises en charge. De plus, les environnements avec ou sans astérisque partagent le même nom, donc \lstinline|\NameTheorem{envname*}{...}| a le même effet que \lstinline|\NameTheorem{envname}{...}| .

\medskip
Ensuite, créez cet environnement de l'une des cinq manières suivantes :
\begin{itemize}
    \item \lstinline|\CreateTheorem*{|\meta{name of environment}\lstinline|}|
        \begin{itemize}
            \item Définir un environnement non numéroté \meta{name of environment}
        \end{itemize}
    \item \lstinline|\CreateTheorem{|\meta{name of environment}\lstinline|}|
        \begin{itemize}
            \item Définir un environnement numéroté \meta{name of environment}, numéroté dans l'ordre 1, 2, 3, \dots
        \end{itemize}
    \item \lstinline|\CreateTheorem{|\meta{name of environment}\lstinline|}[|\meta{numbered like}\lstinline|]|
        \begin{itemize}
            \item Définir un environnement numéroté \meta{name of environment}, qui partage le compteur \meta{numbered like}
        \end{itemize}
    \item \lstinline|\CreateTheorem{|\meta{name of environment}\lstinline|}<|\meta{numbered within}\lstinline|>|
        \begin{itemize}
            \item Définir un environnement numéroté \meta{name of environment}, numéroté dans le compteur \meta{numbered within}
        \end{itemize}
    \item \lstinline|\CreateTheorem{|\meta{name of environment}\lstinline|}(|\meta{existed environment}\lstinline|)|\\
    \lstinline|\CreateTheorem*{|\meta{name of environment}\lstinline|}(|\meta{existed environment}\lstinline|)|
        \begin{itemize}
            \item Identifiez \meta{name of environment} avec \meta{existed environment} ou \meta{existed environment}\lstinline|*|.
            \item Cette méthode est généralement utile dans les deux situations suivantes :
                \begin{enumerate}
                    \item Pour utiliser un nom plus concis. Par exemple, avec \lstinline|\CreateTheorem{thm}(theorem)|, on peut alors utiliser le nom \texttt{thm} pour écrire le théorème.
                    \item Pour supprimer la numérotation de certains environnements. Par exemple, on peut supprimer la numérotation de l'environnement \texttt{remark} avec \lstinline|\CreateTheorem{remark}(remark*)|.
                \end{enumerate}
        \end{itemize}
\end{itemize}

\begin{tip}
    Cette macro utilise la fonctionnalité de \textsf{amsthm} en interne, donc le traditionnel \texttt{theoremstyle} lui est également applicable. Il suffit de déclarer le style avant les définitions pertinentes.
\end{tip}

\NameTheorem[FR]{proofidea}{Idée}
\CreateTheorem*{proofidea*}
\CreateTheorem{proofidea}<subsection>

\bigskip
Voici un exemple. Le code suivant :

\begin{code}
  \NameTheorem[FR]{proofidea}{Idée}
  \CreateTheorem*{proofidea*}
  \CreateTheorem{proofidea}<subsection>
\end{code}

définit un environnement non numéroté \lstinline|proofidea*| et un environnement numéroté \lstinline|proofidea| (numérotés dans la sous-section) respectivement. Ils peuvent être utilisés dans le contexte français. L'effet est le suivant :

\vspace{-0.3\baselineskip}
\begin{proofidea*}
    La environnement \lstinline|proofidea*| .
\end{proofidea*}

\vspace{-\baselineskip}
\begin{proofidea}
    La environnement \lstinline|proofidea| .
\end{proofidea}

\LevelTwoTitle{Draft mark}

Vous pouvez utiliser \lstinline|\dnf| pour marquer la partie inachevée. Par example :
\begin{itemize}
    \item \lstinline|\dnf| ou \lstinline|\dnf<...>|. L'effet est : \dnf~ ou \dnf<...>. \\Le texte à l'intérieur changera en fonction de la langue actuelle. Par exemple, il sera affiché sous la forme \UseOtherLanguage{English}{\dnf} en mode anglais.
\end{itemize}

De même, il y a aussi \lstinline|\needgraph| :
\begin{itemize}
    \item \lstinline|\needgraph| ou \lstinline|\needgraph<...>|. L'effet est : \needgraph ou \needgraph<...>Le texte de l'invite change en fonction de la langue actuelle. Par exemple, en mode anglais, il sera affiché sous la forme \UseOtherLanguage{English}{\needgraph}
\end{itemize}

\LevelTwoTitle{Titre, résumé et mots-clés}

\einfart{} possède à la fois les caractéristiques des classes standard et celles des classes \AmS{}.

Par conséquent, le titre et les informations sur l'auteur peuvent être soit écrits de la manière habituelle, conformément à la classe standard \textsf{article} :

\begin{code}
  \title{(*\meta{title}*)}
  \author{(*\meta{author}*)\thanks{(*\meta{text}*)}}
  \date{(*\meta{date}*)}
  \maketitle
  \begin{abstract}
      (*\meta{abstract}*)
  \end{abstract}
  \begin{keyword}
      (*\meta{keywords}*)
  \end{keyword}
\end{code}

ou écrit à la manière des classes \AmS{} :

\begin{code}
  \title{(*\meta{title}*)}
  \author{(*\meta{author}*)}
  \thanks{(*\meta{text}*)}
  \address{(*\meta{address}*)}
  \email{(*\meta{email}*)}
  \date{(*\meta{date}*)}
  \keywords{(*\meta{keywords}*)}
  \subjclass{(*\meta{subjclass}*)}
  \begin{abstract}
      (*\meta{abstract}*)
  \end{abstract}
  \maketitle
\end{code}

Les informations sur l'auteur peuvent contenir plusieurs groupes, écrits comme suit :

\begin{code}
  \author{(*\meta{author 1}*)}
  \address{(*\meta{address 1}*)}
  \email{(*\meta{email 1}*)}
  \author{(*\meta{author 2}*)}
  \address{(*\meta{address 2}*)}
  \email{(*\meta{email 2}*)}
  ...
\end{code}

Parmi eux, l'ordre mutuel de \lstinline|\address|, \lstinline|\curraddr|, \lstinline|\email| n'est pas important.

\LevelTwoTitle{Divers}

\LevelThreeTitle{Les numéros de ligne}
Les numéros de ligne peuvent être activés et désactivés n'importe où dans votre texte. \lstinline|\linenumbers| est pour activer les numéros de ligne, et \lstinline|\nolinenumbers| est pour les désactiver. Par souci de beauté, le titre, la table des matières, l'index et certains autres éléments ne sont pas numérotés.

\LevelThreeTitle{Les notes de bas de page dans le titre}
Dans \lstinline|\section| ou \lstinline|\subsection| , si vous souhaitez ajouter des notes de bas de page, vous n'avez d'autre choix que :
\begin{itemize}
    \item écrivez d'abord \lstinline|\mbox{\protect\footnotemark}|,
    \item puis ajoutez \lstinline|\footnotetext{...}| après le titre.
\end{itemize}
C'est un inconvénient provoqué par la décoration de soulignement du titre.

\LevelThreeTitle{Les symboles QED}
Puisque la police dans les environnements de type théorème est la même que celle du texte principal, afin d'indiquer où se terminent les environnements, un symbole QED creux \simpleqedsymbol{} est placé à la fin des environnements de type théorème. Cependant, si votre théorème se termine par une équation ou une liste (itemize, énumérer, description, etc.), ce symbole ne peut pas être automatiquement placé à la bonne position. Dans ce cas, vous devez ajouter manuellement un \lstinline|\qedhere| à la fin de votre équation ou la dernière entrée de votre liste pour faire apparaître le symbole QED en fin de ligne.

\LevelOneTitle{Problèmes connus}

\begin{itemize}[itemsep=.6em]
    \item Les paramètres de police ne sont pas encore parfaits.
    \item Comme de nombreuses fonctionnalités sont basées sur la boîte à outils \ProjLib{}, \minimalist{} (et donc \minimart{}, \einfart{} et \minimbook{}, \simplivre{}) hérite de tous ses problèmes. Pour plus de détails, veuillez vous référer à la section « Problèmes connus » de la documentation de \ProjLib{}.
    \item Le mécanisme de gestion des erreurs est incomplet : pas de messages correspondants lorsque certains problèmes surviennent.
    \item Il y a encore beaucoup de choses qui peuvent être optimisées dans le code.
\end{itemize}


\end{document}
\endinput
%%
%% End of file `einfart/einfart-doc-fr.tex'.
