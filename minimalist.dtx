% \iffalse meta-comment
%
% Copyright (C) 2021 by Jinwen XU 
% -------------------------------
% 
% This file may be distributed and/or modified under the conditions of the LaTeX
% Project Public License, either version 1.3c of this license or (at your option)
% any later version. The latest version of this license is in:
%
%    http://www.latex-project.org/lppl.txt
%
% \fi
%
%<*driver>
\ProvidesFile{minimalist.dtx}
%</driver>
\NeedsTeXFormat{LaTeX2e}[2020-10-01]
%
%<*minimart>
\ProvidesClass{minimart}
    [2021/07/30 A simple and clear article style]
\def\minimclass@baseclass{article}
%</minimart>
%
%<*minimbook>
\ProvidesClass{minimbook}
    [2021/07/30 A simple and clear book style]
\def\minimclass@baseclass{book}
%</minimbook>
%
%<*einfart>
\ProvidesClass{einfart}
    [2021/07/30 A simple and clear article style]
\def\minimclass@baseclass{article}
%</einfart>
%
%<*simplivre>
\ProvidesClass{simplivre}
    [2021/07/30 A simple and clear book style]
\def\minimclass@baseclass{book}
%</simplivre>
%
%<*minimalist>
\ProvidesPackage{minimalist}
    [2021/07/30 A simple and clear style for articles and books]
%</minimalist>
%
%# 下面两行引入宏包的代码是所有文件共用的。
%# The following two lines are common to all files.
\RequirePackage{kvoptions}
\RequirePackage{etoolbox}
%
%<*class>
%# 这段代码定义了文档类中的选项。
%# This code defines the options in the document class.
\SetupKeyvalOptions{
    family = @minimclass,
    prefix = @minimclass@,
}
%# 草稿模式或快速模式
%# Draft mode or fast mode
\DeclareBoolOption[false]{draft}
\DeclareBoolOption[false]{fast}
%# 经典样式
%# Classic style
\DeclareBoolOption[false]{classical}
%# 启用首行缩进而不是默认的段间空行
%# Enable paragraph indentation instead of the default parskip
\DeclareBoolOption[false]{useindent}
%# 启用 old-style 数字
%# Enable old-style figures
\DeclareBoolOption[false]{useosf}
%# 几个字体选项
%# Several font options
\DeclareBoolOption[false]{palatino}
\DeclareBoolOption[false]{times}
\DeclareBoolOption[false]{garamond}
\DeclareBoolOption[false]{noto}
\DeclareBoolOption[false]{biolinum}
%# 这一个选项用于表明要采用自定义的字体而不是默认字体
%# This option is used to indicate that one uses a custom font instead of the default font
\DeclareBoolOption[false]{customfont}

%# 纸张尺寸选项 a4paper 和 b5paper
%# Paper size options a4paper and b5paper
\newif\if@minimclass@bfivepaper\@minimclass@bfivepaperfalse
\DeclareVoidOption{b5paper}{\@minimclass@bfivepapertrue}
\newif\if@minimclass@afourpaper\@minimclass@afourpaperfalse
\DeclareVoidOption{a4paper}{\@minimclass@afourpapertrue}

%# 处理选项并载入文档类
%# Process options and load the base document class
\DeclareDefaultOption{\PassOptionsToClass{\CurrentOption}{\minimclass@baseclass}}
\ProcessKeyvalOptions*\relax
\LoadClass{\minimclass@baseclass}

%# 下面的代码设定了一些选项之间的关系
%# The following code sets the relationship between some options
\if@minimclass@draft
    \@minimclass@fasttrue
\fi
\if@minimclass@classical
    \@minimclass@useindenttrue
\fi
\if@minimclass@palatino
    \@minimclass@customfonttrue
\fi
\if@minimclass@times
    \@minimclass@customfonttrue
\fi
\if@minimclass@garamond
    \@minimclass@customfonttrue
\fi
\if@minimclass@noto
    \@minimclass@customfonttrue
\fi
\if@minimclass@biolinum
    \@minimclass@customfonttrue
\fi

%%================================
%% Page layout
%%================================
%# 使用 silence 去除一些警告,主要是这里采用了 total 而不是 margin 的设置方法,因此用户要自行设置页面尺寸时可能会导致此类冲突
%# Use silence to remove some warnings. The main reason is that `total` instead of `margin` is used for setting, so conflict may occur when the user wants to reset the page size
\RequirePackage{silence}
\WarningFilter{geometry}{Over-specification in}
%# 设定页面尺寸
%# Set page size
\RequirePackage[heightrounded]{geometry}
%# 这是默认的尺寸
%# This is the default page size
\geometry{
    papersize={7in,10in},
    total={40em,60em},
    hmarginratio=1:1,
    vmarginratio=1:1,
    footnotesep=2em plus 2pt minus 2pt,
}
%# B5 的尺寸
%# B5 paper size
\if@minimclass@bfivepaper
\geometry{
    b5paper,
    total={40em,59em},
    hmarginratio=1:1,
    vmarginratio=1:1,
    footnotesep=2em plus 2pt minus 2pt,
}
\fi
%# A4 的尺寸
%# A4 paper size
\if@minimclass@afourpaper
\geometry{
    a4paper,
    total={47em,70em},
    hmarginratio=1:1,
    vmarginratio=1:1,
    footnotesep=2em plus 2pt minus 2pt,
}
\fi

%# 这几行代码用于生成快速模式下的 DRAFT 水印
%# To generate DRAFT watermark in fast mode
\if@minimclass@fast
    \PassOptionsToPackage{fast}{minimalist}
    \RequirePackage{draftwatermark}
    \DraftwatermarkOptions{text={\normalfont DRAFT},color=paper!95!-paper}
\fi
%# 这几行代码决定段落的样式
%# To determine the style of the paragraph
\if@minimclass@useindent
    \RequirePackage{indentfirst}
\else
    \AtEndPreamble{\RequirePackage{parskip}}
\fi
%# 载入主样式
%# Load the main style
\RequirePackage{minimalist}

%%================================
%% Fonts
%%================================
%# 下面的字体设置代码用于 minimart 和 minimbook
%# The font setting below is for `minimart` and `minimbook`
%<*minimart|minimbook>
%# 这几行代码在 pdfLaTeX 编译时载入编码设置
%# Load the encoding settings when pdfLaTeX is used
\RequirePackage{iftex}
\ifPDFTeX
\RequirePackage[T1]{fontenc}
\RequirePackage{inputenc}
\fi
%# 这几行代码载入默认字体
%# Load the default font
\if@minimclass@customfont\else
\RequirePackage{mathpazo}
\RequirePackage{newpxtext}
\fi
\RequirePackage{amssymb}
%</minimart|minimbook>
%
%# 下面的字体设置代码用于 einfart 和 simplivre
%# The font setting below is for `einfart` and `simplivre`
%<*einfart|simplivre>
%# 快速模式下的默认数学字体采用 mathpazo,为了让其生效,要放在 fontspec 的前面
%# The default math font in fast mode uses `mathpazo`, in order to make it effective, it must be placed in front of `fontspec`
\if@minimclass@customfont\else
    \if@minimclass@fast
        \RequirePackage{mathpazo}
    \fi
\fi

%# 默认英文字体
%# The default English main font
\PassOptionsToPackage{no-math}{fontspec}
\RequirePackage{fontspec}
\IfFontExistsTF{Palatino Linotype}{%
    \setmainfont{Palatino Linotype}
        [\if@minimclass@useosf Numbers = OldStyle\fi]
}{
    \setmainfont{TeXGyrePagellaX-Regular.otf}[
        BoldFont       = TeXGyrePagellaX-Bold.otf ,
        ItalicFont     = TeXGyrePagellaX-Italic.otf ,
        BoldItalicFont = TeXGyrePagellaX-BoldItalic.otf,
        \if@minimclass@useosf Numbers = OldStyle\fi ]
}
%# 默认英文无衬线字体
%# The default English sans font
    \setsansfont{SourceSansPro-Regular.otf}[
        Scale          = MatchLowercase ,
        BoldFont       = SourceSansPro-Bold.otf ,
        ItalicFont     = SourceSansPro-RegularIt.otf ,
        BoldItalicFont = SourceSansPro-BoldIt.otf ]
%# 默认英文等宽字体
%# The default English mono font
    \setmonofont{NewCMMono10-Regular.otf}[
        Scale          = 1.05 ,
        BoldFont       = NewCMMono10-Book.otf ,
        ItalicFont     = NewCMMono10-Italic.otf ,
        BoldItalicFont = NewCMMono10-BookItalic.otf ]

%# 默认中文字体
%# The default CJK main font
\PassOptionsToPackage{fontset=none,scheme=plain}{ctex}
\RequirePackage{ctex}

\IfFontExistsTF{SourceHanSerifSC-Regular}{%
    \setCJKmainfont{SourceHanSerifSC-Regular}[
        BoldFont       = SourceHanSerifSC-Bold ,
        ItalicFont     = * ,
        BoldItalicFont = SourceHanSerifSC-Bold ,
        SmallCapsFont  = * ]
}{
    \setCJKmainfont{FandolSong-Regular.otf}[
        BoldFont       = FandolSong-Bold.otf ,
        ItalicFont     = FandolKai-Regular.otf ,
        BoldItalicFont = FandolKai-Regular.otf ,
        BoldItalicFeatures = {FakeBold=4} ,
        SmallCapsFont  = * ]
}

%# 默认中文无衬线字体
%# The default CJK sans font
\IfFontExistsTF{SourceHanSansSC-Regular}{%
    \setCJKsansfont{SourceHanSansSC-Regular}[
        BoldFont       = SourceHanSansSC-Bold ,
        ItalicFont     = * ,
        BoldItalicFont = SourceHanSansSC-Bold ,
        SmallCapsFont  = * ]
}{
    \setCJKsansfont{FandolHei-Regular.otf}[
        BoldFont       = FandolHei-Bold.otf ,
        ItalicFont     = * ,
        BoldItalicFont = FandolHei-Bold.otf ,
        SmallCapsFont  = * ]
}

%# 默认中文等宽字体
%# The default CJK mono font
\IfFontExistsTF{SourceHanMonoSC-Regular}{%
    \setCJKmonofont{SourceHanMonoSC-Regular}[
        BoldFont       = SourceHanMonoSC-Bold ,
        ItalicFont     = * ,
        BoldItalicFont = SourceHanMonoSC-Bold ,
        SmallCapsFont  = * ]
}{
    \setCJKmonofont{FandolFang-Regular.otf}[
        BoldFont       = * ,
        BoldFeatures   = {FakeBold=4} ,
        ItalicFont     = * ,
        BoldItalicFont = * ,
        BoldItalicFeatures = {FakeBold=4} ,
        SmallCapsFont  = * ]
}

%# 简体中文字体族
%# SC font family
\if@PJLlang@enable@CN
\IfFontExistsTF{SourceHanSerifSC-Regular}{%
    \setCJKfamilyfont{SCmain}{SourceHanSerifSC-Regular}[
        BoldFont       = SourceHanSerifSC-Bold ,
        ItalicFont     = * ,
        BoldItalicFont = SourceHanSerifSC-Bold ,
        SmallCapsFont  = * ]
}{
    \setCJKfamilyfont{SCmain}{FandolSong-Regular.otf}[
        BoldFont       = FandolSong-Bold.otf ,
        ItalicFont     = FandolKai-Regular.otf ,
        BoldItalicFont = FandolKai-Regular.otf ,
        BoldItalicFeatures = {FakeBold=4} ,
        SmallCapsFont  = * ]
}
\IfFontExistsTF{SourceHanSansSC-Regular}{%
    \setCJKfamilyfont{SCsans}{SourceHanSansSC-Regular}[
        BoldFont       = SourceHanSansSC-Bold ,
        ItalicFont     = * ,
        BoldItalicFont = SourceHanSansSC-Bold ,
        SmallCapsFont  = * ]
}{
    \setCJKfamilyfont{SCsans}{FandolHei-Regular.otf}[
        BoldFont       = FandolHei-Bold.otf ,
        ItalicFont     = * ,
        BoldItalicFont = FandolHei-Bold.otf ,
        SmallCapsFont  = * ]
}
\IfFontExistsTF{SourceHanMonoSC-Regular}{%
    \setCJKfamilyfont{SCmono}{SourceHanMonoSC-Regular}[
        BoldFont       = SourceHanMonoSC-Bold ,
        ItalicFont     = * ,
        BoldItalicFont = SourceHanMonoSC-Bold ,
        SmallCapsFont  = * ]
}{
    \setCJKfamilyfont{SCmono}{FandolFang-Regular.otf}[
        BoldFont       = * ,
        BoldFeatures   = {FakeBold=4} ,
        ItalicFont     = * ,
        BoldItalicFont = * ,
        BoldItalicFeatures = {FakeBold=4} ,
        SmallCapsFont  = * ]
}
\fi

%# 繁体中文字体族
%# TC font family
\if@PJLlang@enable@TC
\IfFontExistsTF{SourceHanSerifTC-Regular}{%
    \setCJKfamilyfont{TCmain}{SourceHanSerifTC-Regular}[
        BoldFont       = SourceHanSerifTC-Bold ,
        ItalicFont     = * ,
        BoldItalicFont = SourceHanSerifTC-Bold ,
        SmallCapsFont  = * ]
}{
    \setCJKfamilyfont{TCmain}{FandolSong-Regular.otf}[
        BoldFont       = FandolSong-Bold.otf ,
        ItalicFont     = FandolKai-Regular.otf ,
        BoldItalicFont = FandolKai-Regular.otf ,
        BoldItalicFeatures = {FakeBold=4} ,
        SmallCapsFont  = * ]
}
\IfFontExistsTF{SourceHanSansTC-Regular}{%
    \setCJKfamilyfont{TCsans}{SourceHanSansTC-Regular}[
        BoldFont       = SourceHanSansTC-Bold ,
        ItalicFont     = * ,
        BoldItalicFont = SourceHanSansTC-Bold ,
        SmallCapsFont  = * ]
}{
    \setCJKfamilyfont{TCsans}{FandolHei-Regular.otf}[
        BoldFont       = FandolHei-Bold.otf ,
        ItalicFont     = * ,
        BoldItalicFont = FandolHei-Bold.otf ,
        SmallCapsFont  = * ]
}
\IfFontExistsTF{SourceHanMonoTC-Regular}{%
    \setCJKfamilyfont{TCmono}{SourceHanMonoTC-Regular}[
        BoldFont       = SourceHanMonoTC-Bold ,
        ItalicFont     = * ,
        BoldItalicFont = SourceHanMonoTC-Bold ,
        SmallCapsFont  = * ]
}{
    \setCJKfamilyfont{TCmono}{FandolFang-Regular.otf}[
        BoldFont       = * ,
        BoldFeatures   = {FakeBold=4} ,
        ItalicFont     = * ,
        BoldItalicFont = * ,
        BoldItalicFeatures = {FakeBold=4} ,
        SmallCapsFont  = * ]
}
\fi

%# 日文字体族
%# JP font family
\if@PJLlang@enable@JP
\IfFontExistsTF{SourceHanSerif-Regular}{%
    \setCJKfamilyfont{JPmain}{SourceHanSerif-Regular}[
        BoldFont       = SourceHanSerif-Bold ,
        ItalicFont     = * ,
        BoldItalicFont = SourceHanSerif-Bold ,
        SmallCapsFont  = * ]
}{
    \setCJKfamilyfont{JPmain}{FandolSong-Regular.otf}[
        BoldFont       = FandolSong-Bold.otf ,
        ItalicFont     = FandolKai-Regular.otf ,
        BoldItalicFont = FandolKai-Regular.otf ,
        BoldItalicFeatures = {FakeBold=4} ,
        SmallCapsFont  = * ]
}
\IfFontExistsTF{SourceHanSans-Regular}{%
    \setCJKfamilyfont{JPsans}{SourceHanSans-Regular}[
        BoldFont       = SourceHanSans-Bold ,
        ItalicFont     = * ,
        BoldItalicFont = SourceHanSans-Bold ,
        SmallCapsFont  = * ]
}{
    \setCJKfamilyfont{JPsans}{FandolHei-Regular.otf}[
        BoldFont       = FandolHei-Bold.otf ,
        ItalicFont     = * ,
        BoldItalicFont = FandolHei-Bold.otf ,
        SmallCapsFont  = * ]
}
\IfFontExistsTF{SourceHanMono-Regular}{%
    \setCJKfamilyfont{JPmono}{SourceHanMono-Regular}[
        BoldFont       = SourceHanMono-Bold ,
        ItalicFont     = * ,
        BoldItalicFont = SourceHanMono-Bold ,
        SmallCapsFont  = * ]
}{
    \setCJKfamilyfont{JPmono}{FandolFang-Regular.otf}[
        BoldFont       = * ,
        BoldFeatures   = {FakeBold=4} ,
        ItalicFont     = * ,
        BoldItalicFont = * ,
        BoldItalicFeatures = {FakeBold=4} ,
        SmallCapsFont  = * ]
}
\fi

%# 选定对应语言的 CJK 字体族
%# Select the CJK font family of the corresponding language
\def\CJKsffamily{}
\def\CJKttfamily{}

\apptocmd{\sffamily}{\CJKsffamily}{}{\FAIL}
\apptocmd{\ttfamily}{\CJKttfamily}{}{\FAIL}

\if@PJLlang@enable@CN
\gappto{\PJLlang@langconfig@chinese}{%
    \def\CJKsffamily{\CJKfamily{SCsans}}%
    \def\CJKttfamily{\CJKfamily{SCmono}}%
    \CJKfamily{SCmain}%
}
\fi

\if@PJLlang@enable@TC
\gappto{\PJLlang@langconfig@tchinese}{%
    \def\CJKsffamily{\CJKfamily{TCsans}}%
    \def\CJKttfamily{\CJKfamily{TCmono}}%
    \CJKfamily{TCmain}%
}
\fi

\if@PJLlang@enable@JP
\gappto{\PJLlang@langconfig@japanese}{%
    \def\CJKsffamily{\CJKfamily{JPsans}}%
    \def\CJKttfamily{\CJKfamily{JPmono}}%
    \CJKfamily{JPmain}%
}
\fi

%# 标题字体:英文时为普通字体,中文时为无衬线字体
%# Titles' font: normal font for English, sans serif font for CJK
\def\minimalist@titlefontshape{\CJKsffamily}

%# 默认数学字体
%# The default math font
\if@minimclass@customfont\else
%## 快速模式下只引用 amssymb 以提供 \mathbb 等字形的定义
%## In fast mode, only load `amssymb` to provide definitions of \mathbb and other shapes
\if@minimclass@fast
\RequirePackage{amssymb}
\else
%## 正常模式下使用 unicode-math
%## Use `unicode-math` in normal mode
\PassOptionsToPackage
    {warnings-off={mathtools-colon,mathtools-overbracket}}{unicode-math}
\RequirePackage{unicode-math}
\unimathsetup{math-style=ISO, partial=upright, nabla=upright}
%### 主字体采用 Asana Math,与 Palatino 字体搭配
%### The base font is Asana Math, which matches with the Palatino font
\setmathfont{Asana-Math.otf}
\IfFontExistsTF{Neo Euler}{%
%###- 下面这段未被采用的代码记录了 Neo Euler 字体的全部支持范围,以供参考
%###- The following unused code records the full range of Neo Euler for reference
% See https://tex.stackexchange.com/a/425887
% \setmathfont{Neo Euler}
%     [range={"0000-"0001,"0020-"007E,
%             "00A0,"00A7-"00A8,"00AC,"00AF,"00B1,"00B4-"00B5,"00B7,
%             "00D7,"00F7,
%             "0131,
%             "0237,"02C6-"02C7,"02D8-"02DA,"02DC,
%             "0300-"030C,"030F,"0311,"0323-"0325,"032E-"0332,"0338,
%             "0391-"0393,"0395-"03A1,"03A3-"03A8,"03B1-"03BB,
%             "03BD-"03C1,"03C3-"03C9,"03D1,"03D5-"03D6,"03F5,
%             "2016,"2018-"2019,"2021,"2026-"202C,"2032-"2037,"2044,
%             "2057,"20D6-"20D7,"20DB-"20DD,"20E1,"20EE-"20EF,
%             "210B-"210C,"210E-"2113,"2118,"211B-"211C,"2126-"2128,
%             "212C-"212D,"2130-"2131,"2133,"2135,"2190-"2199,
%             "21A4,"21A6,"21A9-"21AA,"21BC-"21CC,"21D0-"21D5,
%             "2200,"2202-"2209,"220B-"220C,"220F-"2213,"2215-"221E,
%             "2223,"2225,"2227-"222E,"2234-"2235,"2237-"223D,
%             "2240-"224C,"2260-"2269,"226E-"2279,"2282-"228B,"228E,
%             "2291-"2292,"2295-"2299,"22A2-"22A5,"22C0-"22C5,
%             "22DC-"22DD,"22EF,"22F0-"22F1,
%             "2308-"230B,"2320-"2321,"2329-"232A,"239B-"23AE,
%             "23DC-"23DF,
%             "27E8-"27E9,"27F5-"27FE,"2A0C,"2B1A,
%             "1D400-"1D433,"1D49C,"1D49E-"1D49F,"1D4A2,"1D4A5-"1D4A6,
%             "1D4A9-"1D4AC,"1D4AE-"1D4B5,"1D4D0-"1D4E9,"1D504-"1D505,
%             "1D507-"1D50A,"1D50D-"1D514,"1D516-"1D51C,"1D51E-"1D537,
%             "1D56C-"1D59F,"1D6A8-"1D6B8,"1D6BA-"1D6D2,"1D6D4-"1D6DD,
%             "1D6DF,"1D6E1,"1D7CE-"1D7D7 }]
% \setmathfont[range={up/{latin, Latin, greek, Greek}, 
%                     bfup/{latin, Latin, greek, Greek}, 
%                     frak, bffrak, cal}
%             ]{Neo Euler}
%### 在主字体的基础上替换一些来自 Neo Euler 的符号和字母
%### Replace some symbols and letters from Neo Euler on the basis of the base font
\setmathfont[range={up/{Latin,latin,Greek,greek},
                    bfup/{Latin,latin,Greek,greek},
                    cal, bfcal, frak, bffrak,
                    `(,`),`[,`],`\{,`\},`:,`=,\ne,\equiv,`/,\backslash,
                    \in,\notin,\ni,\subset,\supset,\subseteq,\supseteq,
                    \rightarrow,\leftarrow,\mapsto,\hookrightarrow,\hookleftarrow,
                    \sum,\prod,\coprod,
                    \sqrt,\int,\iint,\iiint,\oint},
              script-features={},
              sscript-features={},
              Scale = 1.05,
            ]{Neo Euler}
}{}
%### 再替换一些来自 Latin Modern 的符号,默认符号有些过粗
%### Replace some symbols from Latin Modern, the default symbols are a bit too thick
\setmathfont[range={\leq,\geq,\ll,\gg,\lll,\ggg,\leqslant,\geqslant},
              Scale = 0.95,
            ]{latinmodern-math.otf}
% \setmathfont[range={bfup/num, it, bfit, scr, bfscr,
%                     sfup, sfit, bfsfup, bfsfit, tt, bfcal}
%             ]{Asana-Math.otf}
%### 最后,使用 TeX Gyre Pagella 的数字,这样能得到更好看的分数线
%### Finally, use numbers from TeX Gyre Pagella to get better fraction line
\setmathfont[range={up/num, bfup/num}
            ]{texgyrepagella-math.otf}

%# 下面这段代码缩减 \mid 和 \nmid 前后的水平距离,默认距离过大
%# The following code reduces the horizontal space before and after \mid and \nmid, the default space is too large
\AtBeginDocument{
\let\mid@original\mid
\renewcommand{\mid}{\hspace{-.2em}\mid@original\hspace{-.2em}}
\let\nmid@original\nmid
\renewcommand{\nmid}{\hspace{-.2em}\nmid@original\hspace{-.2em}}
}

%# 接下来使用 tikz-cd 重定义可伸缩箭头,默认的箭头有些粗细不均
%# The following uses tikz-cd to define retractable arrows to avoid problems with unicode-math.
\RequirePackage{tikz-cd}
\newbox\xratbelow
\newbox\xratabove
% Redefine \xrightarrow[below]{above}
\renewcommand{\xrightarrow}[2][]{%
    \setbox\xratbelow=\hbox{\ensuremath{\scriptstyle #1}}%
    \setbox\xratabove=\hbox{\ensuremath{\scriptstyle #2}}%
    \pgfmathsetlengthmacro{\xratlen}{max(\wd\xratbelow, \wd\xratabove) + .8em}%
    \mathrel{\tikz [->, baseline=-.55ex, every node/.style={inner sep=0pt}]
        \draw (0,0) -- node[below=3pt] {\box\xratbelow}
        node[above=2pt] {\box\xratabove}
        (\xratlen,0) ;}}
% Redefine \xlefttarrow[below]{above}
\renewcommand{\xleftarrow}[2][]{%
    \setbox\xratbelow=\hbox{\ensuremath{\scriptstyle #1}}%
    \setbox\xratabove=\hbox{\ensuremath{\scriptstyle #2}}%
    \pgfmathsetlengthmacro{\xratlen}{max(\wd\xratbelow, \wd\xratabove) + .8em}%
    \mathrel{\tikz [<-, baseline=-.55ex, every node/.style={inner sep=0pt}]
        \draw (0,0) -- node[below=3pt] {\box\xratbelow}
        node[above=2pt] {\box\xratabove}
        (\xratlen,0) ;}}
% Redefine \xleftrightarrow[below]{above}
\renewcommand{\xleftrightarrow}[2][]{%
    \setbox\xratbelow=\hbox{\ensuremath{\scriptstyle #1}}%
    \setbox\xratabove=\hbox{\ensuremath{\scriptstyle #2}}%
    \pgfmathsetlengthmacro{\xratlen}{max(\wd\xratbelow, \wd\xratabove) + .8em}%
    \mathrel{\tikz [<->, baseline=-.55ex, every node/.style={inner sep=0pt}]
        \draw (0,0) -- node[below=3pt] {\box\xratbelow}
        node[above=2pt] {\box\xratabove}
        (\xratlen,0) ;}}
% Redefine \xhookrightarrow[below]{above}, using the hookrightarrow from tikz-cd
\renewcommand{\xhookrightarrow}[2][]{%
    \setbox\xratbelow=\hbox{\ensuremath{\scriptstyle #1}}%
    \setbox\xratabove=\hbox{\ensuremath{\scriptstyle #2}}%
    \pgfmathsetlengthmacro{\xratlen}{max(\wd\xratbelow, \wd\xratabove) + .8em}%
    \mathrel{\tikz [baseline=-.55ex, every node/.style={inner sep=0pt}]
        \draw (0,0) edge[commutative diagrams/hookrightarrow] node[below=3pt] {\box\xratbelow}
        node[above=2pt] {\box\xratabove}
        (\xratlen,0) ;}}
% Redefine \xhooklefttarrow[below]{above}, using the hookleftarrow from tikz-cd
\renewcommand{\xhookleftarrow}[2][]{%
    \setbox\xratbelow=\hbox{\ensuremath{\scriptstyle #1}}%
    \setbox\xratabove=\hbox{\ensuremath{\scriptstyle #2}}%
    \pgfmathsetlengthmacro{\xratlen}{max(\wd\xratbelow, \wd\xratabove) + .8em}%
    \mathrel{\tikz [baseline=-.55ex, every node/.style={inner sep=0pt}]
        \draw (0,0) edge[commutative diagrams/hookleftarrow] node[below=3pt] {\box\xratbelow}
        node[above=2pt] {\box\xratabove}
        (\xratlen,0) ;}}
% Redefine \xmapsto[below]{above}, using the mapsto from tikz-cd
\renewcommand{\xmapsto}[2][]{%
    \setbox\xratbelow=\hbox{\ensuremath{\scriptstyle #1}}%
    \setbox\xratabove=\hbox{\ensuremath{\scriptstyle #2}}%
    \pgfmathsetlengthmacro{\xratlen}{max(\wd\xratbelow, \wd\xratabove) + .8em}%
    \mathrel{\tikz [baseline=-.55ex, every node/.style={inner sep=0pt}]
        \draw (0,0) edge[commutative diagrams/mapsto] node[below=3pt] {\box\xratbelow}
        node[above=2pt] {\box\xratabove}
        (\xratlen,0) ;}}
% Define \xlongequal[below]{above}, using the equal from tikz-cd
\newcommand{\xlongequal}[2][]{%
    \setbox\xratbelow=\hbox{\ensuremath{\scriptstyle #1}}%
    \setbox\xratabove=\hbox{\ensuremath{\scriptstyle #2}}%
    \pgfmathsetlengthmacro{\xratlen}{max(\wd\xratbelow, \wd\xratabove) + .8em}%
    \mathrel{\tikz [baseline=-.55ex, every node/.style={inner sep=0pt}]
        \draw (0,0) edge[commutative diagrams/equal] node[below=3pt] {\box\xratbelow}
        node[above=2pt] {\box\xratabove}
        (\xratlen,0) ;}}
%# 使用 tikz-cd 重新定义一些箭头,默认箭头过粗
%# Redefine \twoheadrightarrow and \twoheadleftarrow, using those from tikz-cd
\AtBeginDocument{
\renewcommand{\twoheadrightarrow}{%
    \mathrel{\tikz [baseline=-.55ex, every node/.style={inner sep=0pt}]
        \draw (0,0) edge[commutative diagrams/twoheadrightarrow] (.8em,0) ;}}
\renewcommand{\twoheadleftarrow}{%
    \mathrel{\tikz [baseline=-.55ex, every node/.style={inner sep=0pt}]
        \draw (0,0) edge[commutative diagrams/twoheadleftarrow] (.8em,0) ;}}
}
\fi
\fi
%</einfart|simplivre>

%# 下面配置字体选项
%# Configure the font options below
\if@minimclass@palatino
    \RequirePackage{amssymb}
    \RequirePackage{mathpazo}
    \PassOptionsToPackage{defaultsups}{newpxtext}
    \RequirePackage{newpxtext}
    % \RequirePackage{newpxmath}
    \let\openbox\relax
    \PJLsetlinespacing{\setstretch{1.07}}
    \if@minimclass@useosf\useosf\fi
\fi

\if@minimclass@times
    \RequirePackage{amssymb}
    \PassOptionsToPackage{defaultsups}{newtxtext}
    \RequirePackage{newtxtext}
    \RequirePackage{newtxmath}
    \let\openbox\relax
    \PJLsetlinespacing{\setstretch{1.00}}
    \if@minimclass@useosf\useosf\fi
\fi

\if@minimclass@garamond
    \RequirePackage{amssymb}
    \if@minimclass@useosf\else
        \PassOptionsToPackage{lining}{ebgaramond}
    \fi
    \RequirePackage{ebgaramond-maths}
    \RequirePackage{ebgaramond}
    % \RequirePackage{garamondlibre}
    \PJLsetlinespacing{\setstretch{1.00}}
\fi

\if@minimclass@noto
    \RequirePackage{amssymb}
    \if@minimclass@useosf
        \PassOptionsToPackage{oldstyle}{notomath}
    \fi
    \RequirePackage{anyfontsize}
    \RequirePackage{notomath}
    \PJLsetlinespacing{\setstretch{1.00}}
\fi

\if@minimclass@biolinum
    \RequirePackage{amssymb}
    % \RequirePackage{mathpazo}
    % \RequirePackage[sansmath]{libertinust1math}
    \ifPDFTeX
        \if@minimclass@useosf
            \PassOptionsToPackage{osf}{biolinum}
        \fi
        \RequirePackage{biolinum}
        \renewcommand*\familydefault{\biolinum@family}
    \else
        \PassOptionsToPackage{no-math}{fontspec}
        \RequirePackage{fontspec}
        \setmainfont{LinBiolinum_R.otf}[
            BoldFont       = LinBiolinum_RB.otf ,
            ItalicFont     = LinBiolinum_RI.otf ,
            BoldItalicFont = LinBiolinum_RBO.otf,
            \if@minimclass@useosf Numbers = OldStyle\fi ]
    \fi
    \PJLsetlinespacing{\setstretch{1.05}}
    \RequirePackage[italic,defaultmathsizes]{mathastext}
\fi

%# 这一行代码会让 ProjLib 中的字体选项失效
%# The following line will invalidate the font options of ProjLib
\def\PJL@font@alreadyset{}

%# 使用 microtype 优化排版细节
%# Use `microtype` to get details better
\PassOptionsToPackage{verbose=silent}{microtype}
\RequirePackage{microtype}
%# 使用 nowidow 以减少换页时的单行
%# Use `nowidow` to reduce single line near the edge
\PassOptionsToPackage{all}{nowidow}
\RequirePackage{nowidow}
%# 使用 embrac 以在 \emph 中获得直立的括号
%# Use `embrac` to get upright brackets in \emph
\RequirePackage{embrac}

%%================================
%% Graphics
%%================================
%# 下面的代码提供了基本的图像和浮动体配置功能
%# The code below provides basic graphic and float configuration
\RequirePackage{graphicx}
\graphicspath{{images/}}
\RequirePackage{wrapfig}
\RequirePackage{float}
\RequirePackage{caption}
\captionsetup{font=small}

%%================================
%% Index
%%================================
% \RequirePackage{imakeidx}
%# 下面的代码用于去除索引中的行号
%# Switch off the line numbers in the index
\AtEndPreamble{
\pretocmd{\printindex}{\LocallyStopLineNumbers}{}{}
\apptocmd{\printindex}{\ResumeLineNumbers}{}{}
}
%</class>
%
%
%<*minimalist>
%# 这段代码定义了宏包中的选项。
%# This code defines the options in the package.
\SetupKeyvalOptions{%
    family = @minimalist,
    prefix = @minimalist@
}
%# 草稿模式或快速模式
%# Draft mode or fast mode
\DeclareBoolOption[false]{draft}
\DeclareBoolOption[false]{fast}
%# 纯文本模式 (不加载定理类环境)
%# Pure text mode (no theorem-like environments loaded)
\DeclareBoolOption[false]{puretext}
\DeclareBoolOption[false]{nothms}
%# 定理类环境不编号
%# No numbering for theorem-like environments
\DeclareBoolOption[false]{nothmnum}
%# 指定如何编号定理类环境
%# Number theorem-like environments With ...
\DeclareStringOption{thmnum}[minimalist@thmnum@default]
%# 引用的名称是否将随当前语言的不同而改变
%# Whether reference name will change with the current language
\DeclareBoolOption[true]{regionalref}
\DeclareBoolOption[false]{originalref}
%# 允许标题中的粗体
%# Allow bold faces in the titles
\DeclareBoolOption[false]{allowbf}
%# 经典样式
%# Classic style
\DeclareBoolOption[false]{classical}
%# subsubsection 标题采用 runin 样式
%# Use `runin` style for subsubsection
\DeclareBoolOption[false]{runin}
%# 处理选项
%# Process options
\ProcessKeyvalOptions*\relax

%# 下面的代码设定了一些选项之间的关系
%# The following code sets the relationship between some options
\if@minimalist@originalref
    \@minimalist@regionalreffalse
\fi
\if@minimalist@draft
  \@minimalist@fasttrue
\fi
\if@minimalist@puretext
  \@minimalist@nothmstrue
\fi

%# 向组件传递一些参数
%# Pass some options to the components
\if@minimalist@fast
    \PassOptionsToPackage{fast}{PJLdraft}
    \PassOptionsToPackage{fast}{PJLthm}
\fi
\expandafter\ifstrempty\expandafter{\@minimalist@thmnum}{}{
    \expandafter\ifstrequal\expandafter{\@minimalist@thmnum}{minimalist@thmnum@default}{
        \PassOptionsToPackage{thmnum}{PJLthm}
    }{
        \PassOptionsToPackage{thmnum=\@minimalist@thmnum}{PJLthm}
    }
}
\if@minimalist@regionalref\else
    \PassOptionsToPackage{originalref}{PJLthm}
\fi

%# 设定带条件的加粗
%# Set the conditional bold face
\if@minimalist@allowbf
    \newcommand{\conditionalbfseries}{\bfseries\colorlet{PJLtempcolor}{.}\color{PJLtempcolor!83!paper}}
\else
    \newcommand{\conditionalbfseries}{}
\fi

%# 判断是文章还是书
%# Article or Book?
\newif\ifIsBook
\ifdefined\c@chapter\IsBooktrue\else\IsBookfalse\fi

%%================================
%% Title fonts
%%================================
\RequirePackage{anyfontsize}

%# 标题共用的字形(在文档类中设置,默认为空)
%# The common font shape of the titles (set in the document class, default is none)
\def\minimalist@titlefontshape{}
% \def\minimalist@titlefontshape{\sffamily}

%# 各级标题的字形
%# The font shape of the titles
\if@minimalist@classical
\newcommand{\partfont}{\conditionalbfseries\minimalist@titlefontshape}
\newcommand{\chapfont}{\conditionalbfseries\minimalist@titlefontshape}
\newcommand{\secfont}{\conditionalbfseries\minimalist@titlefontshape}
\newcommand{\subsecfont}{\conditionalbfseries\minimalist@titlefontshape}
\newcommand{\subsubsecfont}{\conditionalbfseries\minimalist@titlefontshape\itshape}
\else
\newcommand{\partfont}{\conditionalbfseries\minimalist@titlefontshape}
\newcommand{\chapfont}{\conditionalbfseries\minimalist@titlefontshape}
\newcommand{\secfont}{\conditionalbfseries}
\newcommand{\subsecfont}{\conditionalbfseries}
\newcommand{\subsubsecfont}{\conditionalbfseries\itshape}
\fi

%%================================
%% Paper configuration
%%================================
%# 引入 ProjLib 中用于设置页面颜色等参数的模块
%# Introduce the module of ProjLib used to set page color etc.
\RequirePackage{PJLpaper}

%%================================
%% Footer
%%================================
\RequirePackage{geometry}
\RequirePackage{fancyhdr}
\RequirePackage{extramarks}

%# 下面这一行用来修复页面参数被修改时错误的页眉页脚位置
%# The following line is used to fix the wrong header and footer position when the page parameters are modified
\AtEndPreamble{\fancyhfoffset{0pt}}

%# 默认的页眉页脚设置
%# Default header and footer settings
\fancypagestyle{fancy}{
    \fancyhf{}
    \if@twoside
        \fancyfoot[RO]{\small\textcolor{maintext!30!paper}{\lastrightmark}%
            ~~\rlap{\textcolor{maintext!27!paper}{$|$}~~\thepage}}
        \fancyfoot[LE]{\small\leavevmode\llap{\thepage%
            ~~\textcolor{maintext!27!paper}{$|$}}%
            ~~\textcolor{maintext!30!paper}{\lastleftmark}}
    \else
        \fancyfoot[R]{\small\textcolor{maintext!30!paper}{\lastrightmark}%
            ~~\rlap{\textcolor{maintext!27!paper}{$|$}~~\thepage}}
    \fi
    \renewcommand{\headrulewidth}{0pt}
}
\pagestyle{fancy}

%# 第一页的页眉页脚设置
%# First page's header and footer settings
\fancypagestyle{plain}{
    \fancyhf{}
    \if@twoside
        \fancyfoot[RO]{\small%
            ~\rlap{\textcolor{maintext!27!paper}{$|$}~~\thepage}}
        \fancyfoot[LE]{\small\leavevmode\llap{\thepage%
            ~~\textcolor{maintext!27!paper}{$|$}}}
    \else
        \fancyfoot[R]{\small%
            ~\rlap{\textcolor{maintext!27!paper}{$|$}~~\thepage}}
    \fi
    \renewcommand{\headrulewidth}{0pt}
}
\ifbool{IsBook}{
%# 针对书的设定
%# Configure for books
%## part 页的页面及页眉页脚设置
%## Part pages' style and their header and footer settings
    \if@minimalist@fast
    \newcommand{\drawHelpLine}{}
    \else
    \RequirePackage{tikz}
    \usetikzlibrary{calc,shadings}
    \RequirePackage{tikzpagenodes}% For `current page text area`
    \newcommand{\drawHelpLine}{%
        \begin{tikzpicture}[remember picture,overlay]
            \foreach\i in {0,1,...,5}{%
                \fill[opacity=0.12-0.02*\i] 
                    ($(current page text area.north east)
                        +(-\i*0.5em-.025em,-10pt+\i*1.1pt)$) 
                    rectangle ($(current page text area.south east)
                        +(-\i*0.5em+.025em,10pt-\i*1.1pt)$);
                \shade[top color=paper,bottom color=maintext,opacity=0.12-0.02*\i] 
                    ($(current page text area.north east)
                        +(-\i*0.5em-.025em,2pt)$) 
                    rectangle ($(current page text area.north east)
                        +(-\i*0.5em+.025em,-10pt+\i*1.1pt)$);
                \shade[top color=maintext,bottom color=paper,opacity=0.12-0.02*\i] 
                    ($(current page text area.south east)
                        +(-\i*0.5em-.025em,-2pt)$) 
                    rectangle ($(current page text area.south east)
                        +(-\i*0.5em+.025em,10pt-\i*1.1pt)$);
            }
        \end{tikzpicture}%
    }
    \fi
    \fancypagestyle{part}{
        \fancyhf{}
        \renewcommand{\headrulewidth}{0pt}
        \fancyhead[C]{\drawHelpLine}
    }
    \addtolength{\headheight}{20pt}
    \addtolength{\topmargin}{-20pt}
%## mark 的设置
%## Configure the marks
    \if@twoside
        \renewcommand{\chaptermark}[1]{\markboth{\textsc{#1}}{}}
    \else
        \renewcommand{\chaptermark}[1]{\markboth{\textsc{#1}}{\textsc{#1}}}
    \fi
    \renewcommand*{\sectionmark}[1]{%
        \markright{\sec@decochar~\arabic{section}~\sec@decochar~~~#1}}
}{
%# 针对文章的设定
%# Configure for articles
%## mark 的设置
%## Configure the marks
    \if@twoside
        \renewcommand*{\sectionmark}[1]{\markboth{\textsc{#1}}{}}
    \else
        \renewcommand*{\sectionmark}[1]{\markboth{\textsc{#1}}{\textsc{#1}}}
    \fi
}
%
%%================================
%% Languages and line spacing
%%================================
%# 引入 ProjLib 中用于提供多语言设置的模块
%# Introduce the module of ProjLib that offers language-specific configurations
\RequirePackage{PJLlang}
%# 行距设置,中文大约是 1.5 倍行距,英文大约是单倍行距,根据字体变化
%# Line spacing configuration, CJK uses about one-half spacing, and English is single line spacing, depending on the font
% \RequirePackage{setspace}
\PJLsetlinespacing{\setstretch{1.07}}
\PJLsetCJKlinespacing{\onehalfspacing}
% To avoid `Underfull \vbox (badness 10000)`
\raggedbottom

%%================================
%% Line numbers
%%================================
%# 设置行号
%# Line numbers' configuration
\PassOptionsToPackage{pagewise,mathlines}{lineno}
\RequirePackage{lineno}
\renewcommand\linenumberfont{\ttfamily\color{maintext!7!paper}\footnotesize}
\setlength\linenumbersep{1em}

\RequirePackage{mathtools}

%# 对 lineno 的补丁,使得数学公式能够被正确编上行号
%# Patch to lineno, so that equations can be correctly numbered
% From https://tex.stackexchange.com/a/461192
\ifdefined\linenomathpatch\else
% Patch 'normal' math environments:
\newcommand*\linenomathpatch[1]{%
    \cspreto{#1}{\linenomath}%
    \cspreto{#1*}{\linenomath}%
    \csappto{end#1}{\endlinenomath}%
    \csappto{end#1*}{\endlinenomath}%
}
% Patch AMS math environments:
\newcommand*\linenomathpatchAMS[1]{%
    \cspreto{#1}{\linenomathAMS}%
    \cspreto{#1*}{\linenomathAMS}%
    \csappto{end#1}{\endlinenomath}%
    \csappto{end#1*}{\endlinenomath}%
}
% Define \linenomathAMS depending on whether 'mathlines' option is provided
\expandafter\ifx\linenomath\linenomathWithnumbers
    \let\linenomathAMS\linenomathWithnumbers
% The following line gets rid of an extra line numbers at the bottom:
    \patchcmd\linenomathAMS{\advance\postdisplaypenalty\linenopenalty}{}{}{}
\else
    \let\linenomathAMS\linenomathNonumbers
\fi

\linenomathpatch{equation}
\linenomathpatchAMS{gather}
\linenomathpatchAMS{multline}
\linenomathpatchAMS{align}
\linenomathpatchAMS{alignat}
\linenomathpatchAMS{flalign}
\fi

%# 用于记录是否计量行号的 bool 变量
%# Record whether linenumber has turned on
\newif\ifLNturnsON
\def\LocallyStopLineNumbers{\LNturnsONfalse%
    \ifLineNumbers\LNturnsONtrue\fi\nolinenumbers}
\def\ResumeLineNumbers{\ifLNturnsON\linenumbers\fi}

%# 下面的代码用于去除目录中的行号
%# Switch off the line numbers in the TOC
\pretocmd{\tableofcontents}{\LocallyStopLineNumbers}{}{\FAIL}
\apptocmd{\tableofcontents}{\ResumeLineNumbers}{}{\FAIL}
%# 下面的代码用于去除参考文献中的行号
%# Switch off the line numbers in the bibliography
\pretocmd{\thebibliography}{\LocallyStopLineNumbers}{}{\FAIL}
\apptocmd{\endthebibliography}{\ResumeLineNumbers}{}{\FAIL}

%%================================
%% Title format
%%================================
\RequirePackage[explicit,newparttoc]{titlesec}
\PassOptionsToPackage{normalem}{ulem}
\RequirePackage{ulem}

%# 设定 CJK 的 part 标题本地化(babel 目前不包含这个设置)
%# Set CJK's part title localization (`babel` currently does not include this setting)
\newcommand{\partstring}{\MakeUppercase{{\partname~\protect\thepart}}}
\gappto{\PJLlang@langconfig@common}{%
\renewcommand{\partstring}{\MakeUppercase{{\partname~\protect\thepart}}}%
}
\gappto{\PJLlang@langconfig@chinese}{%
\renewcommand{\partstring}{第~\thepart~部分}%
}
\gappto{\PJLlang@langconfig@tchinese}{%
\renewcommand{\partstring}{第~\thepart~部分}%
}
\gappto{\PJLlang@langconfig@japanese}{%
\renewcommand{\partstring}{第~\thepart~部}%
}

\ifbool{IsBook}{
%# 针对书的标题样式设定
%# Configure the title formats for books
    %% Part
    \titleclass{\part}{top} % make part like a chapter
    \titleformat{\part}[display]
        {\thispagestyle{part}%
        \LocallyStopLineNumbers%
        \partfont\filleft}
        {\partstring}
        {1em}
        {\fontsize{20}{0}\selectfont\MakeUppercase{#1}}
        [\ResumeLineNumbers]
    \titleformat{name=\part,numberless}[display]
        {\thispagestyle{part}%
        \LocallyStopLineNumbers%
        % \phantomsection\addcontentsline{toc}{part}{#1}%
        \partfont\filleft}
        {\phantom{\MakeUppercase{\partname}}}
        {1em}
        {\fontsize{20}{0}\selectfont\MakeUppercase{#1}}
        [\ResumeLineNumbers]
    \titlespacing*{\part}{0pt}{5em}{6em}
    %% Text after part
    \newcommand{\parttext}[1]{%
    \vfill%
    \LocallyStopLineNumbers%
    \begin{flushright}%
        \begin{minipage}{0.833\textwidth}%
            \color{maintext!80!paper}\raggedleft#1%
        \end{minipage}%
    \end{flushright}%
    \ResumeLineNumbers%
    \vfill\vfill%
    \cleardoublepage%
    }
    
    %% Chapter
    \titleformat{\chapter}
        {\thispagestyle{fancy}%
        \LocallyStopLineNumbers%
        \color{maintext!80!paper}\chapfont\fontsize{16}{0}\selectfont}{}{0em}
        {\rlap{\hspace*{-.5em}{\color{maintext!12!paper}%
            \fontsize{80}{0}\selectfont\raisebox{-7pt}{\thechapter}}}#1}
        [\ResumeLineNumbers]
    \titleformat{name=\chapter,numberless}
        {\thispagestyle{fancy}%
        \LocallyStopLineNumbers%
        % \phantomsection\addcontentsline{toc}{chapter}{#1}%
        \color{maintext!80!paper}\chapfont\fontsize{16}{0}\selectfont}{}{0em}
        {\rlap{\hspace*{-.5em}{\color{maintext!12!paper}%
            \fontsize{80}{0}\selectfont\normalfont\raisebox{-7pt}{*}}}#1}
        [\ResumeLineNumbers]
}{
%# 针对文章的标题样式设定
%# Configure the title formats for articles
    %% Part
    \titleformat{\part}[display]
        {\LocallyStopLineNumbers%
        \partfont\filleft}
        {\partstring}
        {.3em}
        {\fontsize{16}{0}\selectfont\MakeUppercase{#1}}
        [\ResumeLineNumbers]
    \titleformat{name=\part,numberless}[display]
        {\LocallyStopLineNumbers%
        % \phantomsection\addcontentsline{toc}{part}{#1}%
        \partfont\filleft}
        {\phantom{\MakeUppercase{\partname}}}
        {.3em}
        {\fontsize{16}{0}\selectfont\MakeUppercase{#1}}
        [\ResumeLineNumbers]
    %% Text after part
    \newcommand{\parttext}[1]{%
        \LocallyStopLineNumbers%
        \begin{flushright}%
            \begin{minipage}{0.833\textwidth}%
                \color{maintext!80!paper}\raggedleft#1%
            \end{minipage}%
        \end{flushright}%
        \ResumeLineNumbers%
    }
}

%% Section
\if@minimalist@classical
\newcommand\sec@decochar{}
\titleformat{\section}
    {\LocallyStopLineNumbers%
    \secfont\centering}
    {\thesection}{.75em}
    {#1}
    [\ResumeLineNumbers]
\else
\renewcommand\thesection{\arabic{section}}
\newcommand\seculine{\bgroup\markoverwith{\color{maintext!27!paper}%
    \rule[-0.9ex]{2pt}{.6pt}\hspace{-2pt}\rule[-1.2ex]{2pt}{.6pt}}\ULon}
\newcommand\sec@decochar{\raisebox{.03em}{\normalfont/}}
\titleformat{\section}
    {\LocallyStopLineNumbers%
    \secfont\centering}{}{0em}
    {{\small\textcolor{maintext!27!paper}{\footnotesize\sec@decochar}%
        \,\,\textcolor{maintext!90!paper}{\conditionalbfseries\arabic{section}}%
        \,\,\textcolor{maintext!27!paper}{\footnotesize\sec@decochar}}\\
        \seculine{#1}}
    [\ResumeLineNumbers]
\titleformat{name=\section,numberless}
    {\LocallyStopLineNumbers%
    % \phantomsection\addcontentsline{toc}{section}{#1}%
    \secfont\centering}{}{0em}
    {\seculine{#1}}
    [\ResumeLineNumbers]
\fi

%% Subsection
\if@minimalist@classical
\titleformat{\subsection}
    {\LocallyStopLineNumbers%
    \subsecfont}
    {\thesubsection}{.75em}
    {#1}
    [\ResumeLineNumbers]
\else
\renewcommand\thesubsection{%
    \ifnum\c@section=0\else\arabic{section}.\fi\arabic{subsection}}
\newcommand\subseculine{\bgroup\markoverwith{\color{maintext!27!paper}%
    \rule[-1ex]{2pt}{.75pt}}\ULon}
\titleformat{\subsection}
    {\LocallyStopLineNumbers%
    \subsecfont}{}{0em}
    {\subseculine{\thesubsection~\textcolor{maintext!27!paper}{$|$}~#1}}
    [\ResumeLineNumbers]
\titleformat{name=\subsection,numberless}
    {\LocallyStopLineNumbers%
    \subsecfont}{}{0em}
    {\subseculine{#1}}
    [\ResumeLineNumbers]
\fi

%% Subsubsection
\if@minimalist@runin
\titleformat{\subsubsection}[runin]
    {\color{maintext!70!paper}\subsubsecfont}{\thesubsubsection}{.5em}
    {#1.}[\hspace*{.3em}]
\else
\titleformat{\subsubsection}
    {\LocallyStopLineNumbers%
    \color{maintext!70!paper}\subsubsecfont}{\thesubsubsection}{.5em}
    {#1}
    [\ResumeLineNumbers]
\fi

%% Paragraph
\titleformat{\paragraph}[runin]
    {\scshape}{\theparagraph}{1em}{#1}

\titlespacing{\section}{0pt}{\baselineskip}{.6\baselineskip}
\titlespacing{\subsection}{0pt}{.75\baselineskip}{.4\baselineskip}
\titlespacing{\subsubsection}{0pt}{.5\baselineskip}{.2\baselineskip}

%%================================
%% ToC format
%%================================
\RequirePackage{titletoc}
\titlecontents{part}
    [0em]
    {\addvspace{1.5pc}\filcenter\normalfont}
    {\thecontentslabel\nopagebreak\\\nopagebreak\uppercase}
    {}
    {} % without page number
    [\addvspace{.5pc}]
\ifbool{IsBook}{
%# 针对书的目录样式设定
%# Configure the TOC style for books
    \titlecontents{chapter}
        [2em] % i.e., 0em (part) + 2em
        {\addvspace{.5pc}\normalfont}
        {\contentslabel{2em}}
        {\hspace*{-2em}}
        {\titlerule*[1em]{\textcolor{maintext!15!paper}{.}}\contentspage}
    \titlecontents{section}
        [4em] % i.e., 2em (chapter) + 2em
        {\normalfont}
        {\contentslabel[\textcolor{maintext!27!paper}{\small\sec@decochar}\,\textcolor{maintext!90!paper}{\thecontentslabel}\,\textcolor{maintext!27!paper}{\small\sec@decochar}]{2em}}
        {\hspace*{-2em}}
        {\titlerule*[1em]{\textcolor{maintext!15!paper}{.}}\contentspage}
    \titlecontents{subsection}
        [6.5em] % i.e., 4em (section) + 2.5em
        {\normalfont}
        {\contentslabel{2.25em}}
        {\hspace*{-2.25em}}
        {\titlerule*[1em]{\textcolor{maintext!15!paper}{.}}\contentspage}
    \titlecontents{subsubsection}
        [8.5em] % i.e., 6.5em (subsection) + 3em
        {\normalfont}
        {\contentslabel{2.75em}}
        {\hspace*{-2.75em}}
        {\titlerule*[1em]{\textcolor{maintext!15!paper}{.}}\contentspage}
}{
%# 针对文章的目录样式设定
%# Configure the TOC style for articles
    \titlecontents{section}
        [2em] % i.e., 0em (part) + 2em
        {\normalfont}
        {\contentslabel{1.75em}}
        {\hspace*{-1.75em}}
        {\titlerule*[1em]{\textcolor{maintext!15!paper}{.}}\contentspage}
    \titlecontents{subsection}
        [4.5em] % i.e., 2em (section) + 2.5em
        {\normalfont}
        {\contentslabel{2.25em}}
        {\hspace*{-2.25em}}
        {\titlerule*[1em]{\textcolor{maintext!15!paper}{.}}\contentspage}
    \titlecontents{subsubsection}
        [7.5em] % i.e., 4.5em (subsection) + 3em
        {\normalfont}
        {\contentslabel{2.75em}}
        {\hspace*{-2.75em}}
        {\titlerule*[1em]{\textcolor{maintext!15!paper}{.}}\contentspage}
}

%%================================
%% Lists
%%================================
%# 列表样式设定
%# Configure the style of lists
\RequirePackage{enumitem}
\setlist{noitemsep}
% \setlist[1]{labelindent=\parindent} % < Usually a good idea
\setlist[enumerate]{labelsep=*, leftmargin=*}
\setlist[enumerate,1]{label = \arabic*.,
    ref = \arabic*}
\setlist[enumerate,2]{label = \emph{\alph*}),
    ref = \theenumi.\emph{\alph*}}
\setlist[enumerate,3]{label = (\roman*),
    ref = \theenumii.\roman*}
\setlist[description]{font=\normalfont\conditionalbfseries}

\setlist[itemize]{leftmargin=*}
\gappto{\PJLlang@langconfig@common}{\setlist[itemize,1]{label=\textcolor{maintext!25}{$\bullet$}}}
\gappto{\PJLlang@langconfig@french}{\setlist[itemize,1]{label=\textcolor{maintext!50}{---}}}
\setlist[itemize,2]{label=\textcolor{maintext!27}{--}}
\setlist[itemize,3]{label=\textcolor{maintext!27}{\texttt{*}}}

%%================================
%% Blank page
%%================================
%# 空白页样式
%# The blank page style
\newcommand{\blinkpagetext}{This page is intentionally left blank}
\renewcommand{\cleardoublepage}{\relax
    \clearpage
    \if@twoside\ifodd\c@page\relax\else
    \thispagestyle{empty}
    \AddToHookNext{shipout/background}
      {% 
       \put(0.5\paperwidth,-0.5\paperheight){%
       \makebox[0pt]{\large\color{maintext!10!paper}\blinkpagetext}}}
    \null\newpage\fi\fi}

%%================================
%% Draft mark
%%================================
%# 引入 ProjLib 中用于提供草稿阶段指令的模块
%# Introduce the module of ProjLib that offers macros for draft stage
\RequirePackage{PJLdraft}

%%================================
%% Theorems
%%================================
\RequirePackage{amsthm}
\if@minimalist@classical
%# 经典模式的定理样式
%# The theorem style for classical mode
% Style #1
% \newtheoremstyle{simple}%
%     {}{}%
%     {\normalfont}{}%
%     {\normalfont}{}%
%     {0pt}%
%     {\thmname{\textsc{#1}}\thmnumber{ #2}\hspace{.4em}%
%         \ifstrequal{#3}{}{\@empty\textcolor{maintext!27!paper}{$-$}\hspace{.4em}}%
%         {\color{maintext!50!paper}\thmnote{\ensuremath{(\text{#3})}~~}}}
%
% Style #2
% \newcommand\thmuline{\bgroup\markoverwith{\color{maintext!27!paper}%
%     \rule[-.6ex]{2pt}{.75pt}}\ULon}
% \newtheoremstyle{simple}%
%     {}{}%
%     {\normalfont}{}%
%     {\normalfont}{}%
%     {0pt}%
%     {\thmuline{\thmname{\textsc{#1}}\thmnumber{ #2}}\hspace{.4em}%
%         {\color{maintext!50!paper}\thmnote{\ensuremath{(\text{#3})}~~}}}
%
% Style #3
    \newtheoremstyle{simple}%
        {}{}%
        {\normalfont}{}%
        {\normalfont}{}%
        {0pt}%
        {{\conditionalbfseries\minimalist@titlefontshape\thmname{#1}\thmnumber{ #2}}%
            {\color{maintext!50!paper}\thmnote{\hspace{.4em}\ensuremath{(\text{#3})}}}~~{\normalfont---}~~}
\else
%# 正常模式的定理样式
%# The theorem style for normal mode
    \def\simpleqedsymbol{%
        \makebox[1em]{\rlap{\textcolor{maintext!12!paper}{\rule[-0.1em]{.95em}{.95em}}}{\kern.07em\raisebox{.07em}{\textcolor{paper}{\rule[-0.1em]{.81em}{.81em}}}\kern.07em}}}
    \newtheoremstyle{simple}%
        {}{}%
        {\normalfont}{}%
        {\normalfont}{}%
        {0pt}%
        {\global\let\qedsymbol\simpleqedsymbol%
            {\conditionalbfseries\thmname{#1}\thmnumber{ #2}}\hspace{.4em}%
            \textcolor{maintext!27!paper}{$|$}\hspace{.4em}%
            \color{maintext!50!paper}\thmnote{\ensuremath{(\text{#3})}~~}%
            \pushQED{\qed}%
        }
    \def\@endtheorem{\global\let\qedsymbol\simpleqedsymbol%
        \popQED\endtrivlist\@endpefalse
        \global\let\qedsymbol\qedsymbolOriginal}
\fi

%# 灰色实心的 QED 符号
%# A grey solid QED symbol
\renewcommand{\qedsymbol}{%
    \makebox[1em]{\color{maintext!27!paper}\rule[-0.1em]{.95em}{.95em}}}
\let\qedsymbolOriginal\qedsymbol

%# 超链接的设置
%# The configuration of hyperlinks
\if@minimalist@fast
    \providecommand{\phantomsection}{}
    \RequirePackage{url}
    \newcommand{\href}[2]{#2}
\else
    \PassOptionsToPackage{hidelinks,linktoc=all}{hyperref}
% To solve `Difference between bookmark levels is greater than one`
    \RequirePackage{bookmark}
    \RequirePackage{hyperref}
\fi

%# 引入 ProjLib 中用于设置定理类环境的模块
%# Introduce the module of ProjLib that offers the configuration of theorem-like environments
% Should be placed after "hyperref"
\PassOptionsToPackage{nothms}{PJLthm}
\RequirePackage{PJLthm}

%# 英文定理标题的样式
%# Adjust English theorems names
\pretocmd{\theoremnameEN}{\textsc}{}{\FAIL}
\pretocmd{\lemmanameEN}{\textsc}{}{\FAIL}
\pretocmd{\propositionnameEN}{\textsc}{}{\FAIL}
\pretocmd{\corollarynameEN}{\textsc}{}{\FAIL}
\pretocmd{\propertynameEN}{\textsc}{}{\FAIL}
\pretocmd{\factnameEN}{\textsc}{}{\FAIL}
\pretocmd{\conjecturenameEN}{\textsc}{}{\FAIL}
\pretocmd{\definitionnameEN}{\textsc}{}{\FAIL}
\pretocmd{\axiomnameEN}{\textsc}{}{\FAIL}
\pretocmd{\assumptionnameEN}{\textsc}{}{\FAIL}
\pretocmd{\conventionnameEN}{\textsc}{}{\FAIL}
\pretocmd{\hypothesisnameEN}{\textsc}{}{\FAIL}
\pretocmd{\notationnameEN}{\textsc}{}{\FAIL}
\pretocmd{\examplenameEN}{\textsc}{}{\FAIL}
\pretocmd{\problemnameEN}{\textsc}{}{\FAIL}
\pretocmd{\questionnameEN}{\textsc}{}{\FAIL}
\pretocmd{\exercisenameEN}{\textsc}{}{\FAIL}
\pretocmd{\remarknameEN}{\itshape}{}{\FAIL}
\pretocmd{\observationnameEN}{\itshape}{}{\FAIL}

%# 法语定理标题的样式
%# Adjust French theorems names
\pretocmd{\theoremnameFR}{\textsc}{}{\FAIL}
\pretocmd{\lemmanameFR}{\textsc}{}{\FAIL}
\pretocmd{\propositionnameFR}{\textsc}{}{\FAIL}
\pretocmd{\corollarynameFR}{\textsc}{}{\FAIL}
\pretocmd{\propertynameFR}{\textsc}{}{\FAIL}
\pretocmd{\factnameFR}{\textsc}{}{\FAIL}
\pretocmd{\conjecturenameFR}{\textsc}{}{\FAIL}
\pretocmd{\definitionnameFR}{\textsc}{}{\FAIL}
\pretocmd{\axiomnameFR}{\textsc}{}{\FAIL}
\pretocmd{\assumptionnameFR}{\textsc}{}{\FAIL}
\pretocmd{\conventionnameFR}{\textsc}{}{\FAIL}
\pretocmd{\hypothesisnameFR}{\textsc}{}{\FAIL}
\pretocmd{\notationnameFR}{\textsc}{}{\FAIL}
\pretocmd{\examplenameFR}{\textsc}{}{\FAIL}
\pretocmd{\problemnameFR}{\textsc}{}{\FAIL}
\pretocmd{\questionnameFR}{\textsc}{}{\FAIL}
\pretocmd{\exercisenameFR}{\textsc}{}{\FAIL}
\pretocmd{\remarknameFR}{\itshape}{}{\FAIL}
\pretocmd{\observationnameFR}{\itshape}{}{\FAIL}

%# 德语定理标题的样式
%# Adjust German theorems names
\pretocmd{\theoremnameDE}{\textsc}{}{\FAIL}
\pretocmd{\lemmanameDE}{\textsc}{}{\FAIL}
\pretocmd{\propositionnameDE}{\textsc}{}{\FAIL}
\pretocmd{\corollarynameDE}{\textsc}{}{\FAIL}
\pretocmd{\propertynameDE}{\textsc}{}{\FAIL}
\pretocmd{\factnameDE}{\textsc}{}{\FAIL}
\pretocmd{\conjecturenameDE}{\textsc}{}{\FAIL}
\pretocmd{\definitionnameDE}{\textsc}{}{\FAIL}
\pretocmd{\axiomnameDE}{\textsc}{}{\FAIL}
\pretocmd{\assumptionnameDE}{\textsc}{}{\FAIL}
\pretocmd{\conventionnameDE}{\textsc}{}{\FAIL}
\pretocmd{\hypothesisnameDE}{\textsc}{}{\FAIL}
\pretocmd{\notationnameDE}{\textsc}{}{\FAIL}
\pretocmd{\examplenameDE}{\textsc}{}{\FAIL}
\pretocmd{\problemnameDE}{\textsc}{}{\FAIL}
\pretocmd{\questionnameDE}{\textsc}{}{\FAIL}
\pretocmd{\exercisenameDE}{\textsc}{}{\FAIL}
\pretocmd{\remarknameDE}{\itshape}{}{\FAIL}
\pretocmd{\observationnameDE}{\itshape}{}{\FAIL}

%# 意大利语定理标题的样式
%# Adjust Italian theorems names
\pretocmd{\theoremnameIT}{\textsc}{}{\FAIL}
\pretocmd{\lemmanameIT}{\textsc}{}{\FAIL}
\pretocmd{\propositionnameIT}{\textsc}{}{\FAIL}
\pretocmd{\corollarynameIT}{\textsc}{}{\FAIL}
\pretocmd{\propertynameIT}{\textsc}{}{\FAIL}
\pretocmd{\factnameIT}{\textsc}{}{\FAIL}
\pretocmd{\conjecturenameIT}{\textsc}{}{\FAIL}
\pretocmd{\definitionnameIT}{\textsc}{}{\FAIL}
\pretocmd{\axiomnameIT}{\textsc}{}{\FAIL}
\pretocmd{\assumptionnameIT}{\textsc}{}{\FAIL}
\pretocmd{\conventionnameIT}{\textsc}{}{\FAIL}
\pretocmd{\hypothesisnameIT}{\textsc}{}{\FAIL}
\pretocmd{\notationnameIT}{\textsc}{}{\FAIL}
\pretocmd{\examplenameIT}{\textsc}{}{\FAIL}
\pretocmd{\problemnameIT}{\textsc}{}{\FAIL}
\pretocmd{\questionnameIT}{\textsc}{}{\FAIL}
\pretocmd{\exercisenameIT}{\textsc}{}{\FAIL}
\pretocmd{\remarknameIT}{\itshape}{}{\FAIL}
\pretocmd{\observationnameIT}{\itshape}{}{\FAIL}

%# 葡萄牙语定理标题的样式
%# Adjust Portuguese theorems names
\pretocmd{\theoremnamePT}{\textsc}{}{\FAIL}
\pretocmd{\lemmanamePT}{\textsc}{}{\FAIL}
\pretocmd{\propositionnamePT}{\textsc}{}{\FAIL}
\pretocmd{\corollarynamePT}{\textsc}{}{\FAIL}
\pretocmd{\propertynamePT}{\textsc}{}{\FAIL}
\pretocmd{\factnamePT}{\textsc}{}{\FAIL}
\pretocmd{\conjecturenamePT}{\textsc}{}{\FAIL}
\pretocmd{\definitionnamePT}{\textsc}{}{\FAIL}
\pretocmd{\axiomnamePT}{\textsc}{}{\FAIL}
\pretocmd{\assumptionnamePT}{\textsc}{}{\FAIL}
\pretocmd{\conventionnamePT}{\textsc}{}{\FAIL}
\pretocmd{\hypothesisnamePT}{\textsc}{}{\FAIL}
\pretocmd{\notationnamePT}{\textsc}{}{\FAIL}
\pretocmd{\examplenamePT}{\textsc}{}{\FAIL}
\pretocmd{\problemnamePT}{\textsc}{}{\FAIL}
\pretocmd{\questionnamePT}{\textsc}{}{\FAIL}
\pretocmd{\exercisenamePT}{\textsc}{}{\FAIL}
\pretocmd{\remarknamePT}{\itshape}{}{\FAIL}
\pretocmd{\observationnamePT}{\itshape}{}{\FAIL}

%# 巴西葡萄牙语定理标题的样式
%# Adjust Brazilian theorems names
\pretocmd{\theoremnameBR}{\textsc}{}{\FAIL}
\pretocmd{\lemmanameBR}{\textsc}{}{\FAIL}
\pretocmd{\propositionnameBR}{\textsc}{}{\FAIL}
\pretocmd{\corollarynameBR}{\textsc}{}{\FAIL}
\pretocmd{\propertynameBR}{\textsc}{}{\FAIL}
\pretocmd{\factnameBR}{\textsc}{}{\FAIL}
\pretocmd{\conjecturenameBR}{\textsc}{}{\FAIL}
\pretocmd{\definitionnameBR}{\textsc}{}{\FAIL}
\pretocmd{\axiomnameBR}{\textsc}{}{\FAIL}
\pretocmd{\assumptionnameBR}{\textsc}{}{\FAIL}
\pretocmd{\conventionnameBR}{\textsc}{}{\FAIL}
\pretocmd{\hypothesisnameBR}{\textsc}{}{\FAIL}
\pretocmd{\notationnameBR}{\textsc}{}{\FAIL}
\pretocmd{\examplenameBR}{\textsc}{}{\FAIL}
\pretocmd{\problemnameBR}{\textsc}{}{\FAIL}
\pretocmd{\questionnameBR}{\textsc}{}{\FAIL}
\pretocmd{\exercisenameBR}{\textsc}{}{\FAIL}
\pretocmd{\remarknameBR}{\itshape}{}{\FAIL}
\pretocmd{\observationnameBR}{\itshape}{}{\FAIL}

%# 西班牙语定理标题的样式
%# Adjust Spanish theorems names
\pretocmd{\theoremnameES}{\textsc}{}{\FAIL}
\pretocmd{\lemmanameES}{\textsc}{}{\FAIL}
\pretocmd{\propositionnameES}{\textsc}{}{\FAIL}
\pretocmd{\corollarynameES}{\textsc}{}{\FAIL}
\pretocmd{\propertynameES}{\textsc}{}{\FAIL}
\pretocmd{\factnameES}{\textsc}{}{\FAIL}
\pretocmd{\conjecturenameES}{\textsc}{}{\FAIL}
\pretocmd{\definitionnameES}{\textsc}{}{\FAIL}
\pretocmd{\axiomnameES}{\textsc}{}{\FAIL}
\pretocmd{\assumptionnameES}{\textsc}{}{\FAIL}
\pretocmd{\conventionnameES}{\textsc}{}{\FAIL}
\pretocmd{\hypothesisnameES}{\textsc}{}{\FAIL}
\pretocmd{\notationnameES}{\textsc}{}{\FAIL}
\pretocmd{\examplenameES}{\textsc}{}{\FAIL}
\pretocmd{\problemnameES}{\textsc}{}{\FAIL}
\pretocmd{\questionnameES}{\textsc}{}{\FAIL}
\pretocmd{\exercisenameES}{\textsc}{}{\FAIL}
\pretocmd{\remarknameES}{\itshape}{}{\FAIL}
\pretocmd{\observationnameES}{\itshape}{}{\FAIL}

%# 判断最高级别的标题是 section 还是 chapter
%# Detect the highest level of title: section or chapter?
% \ifbool{IsBook}{
%     \newaliascnt{highest}{chapter}
% }{
%     \newaliascnt{highest}{section}
% }
%# 按照 thmnum 决定定理类环境如何编号
%# Number the theorem-like environment with respect to thmnum
% \RenewDocumentCommand{\CreateTheorem@thmnum}{m}{
%     \expandafter\ifstrempty\expandafter{\@minimalist@thmnum}{
%         \CreateTheorem{#1}<highest>
%     }{
%         \expandafter\ifstrequal\expandafter{\@minimalist@thmnum}{minimalist@thmnum@default}{
%             \CreateTheorem{#1}
%         }{
%             \CreateTheorem{#1}<\@minimalist@thmnum>
%         }
%     }
% }
%# 建立定理类环境
%# Set-up theorem-like environments
\AtEndPreamble{
\theoremstyle{simple}
\if@minimalist@nothms\else
\if@minimalist@nothmnum
\CreateTheorem*{theorem}
\CreateTheorem*{lemma}
\CreateTheorem*{proposition}
\CreateTheorem*{corollary}
\CreateTheorem*{definition-proposition}
\CreateTheorem*{definition-theorem}
\CreateTheorem*{property}
\CreateTheorem*{fact}
\CreateTheorem*{conjecture}
\else
\CreateTheorem@thmnum{theorem}
\CreateTheorem{lemma}[theorem]
\CreateTheorem{proposition}[theorem]
\CreateTheorem{corollary}[theorem]
\CreateTheorem{definition-proposition}[theorem]
\CreateTheorem{definition-theorem}[theorem]
\CreateTheorem{property}[theorem]
\CreateTheorem{fact}[theorem]
\CreateTheorem{conjecture}[theorem]
\fi
\CreateTheorem*{theorem*}
\CreateTheorem*{lemma*}
\CreateTheorem*{proposition*}
\CreateTheorem*{corollary*}
\CreateTheorem*{definition-proposition*}
\CreateTheorem*{definition-theorem*}
\CreateTheorem*{property*}
\CreateTheorem*{fact*}
\CreateTheorem*{conjecture*}
%
\if@minimalist@nothmnum
\CreateTheorem*{definition}
\CreateTheorem*{axiom}
\CreateTheorem*{assumption}
\CreateTheorem*{convention}
\CreateTheorem*{hypothesis}
\CreateTheorem*{notation}
\CreateTheorem*{example}
\CreateTheorem*{problem}
\CreateTheorem*{question}
\CreateTheorem*{exercise}
\else
\CreateTheorem{definition}[theorem]
\CreateTheorem{axiom}[theorem]
\CreateTheorem{assumption}[theorem]
\CreateTheorem{convention}[theorem]
\CreateTheorem{hypothesis}[theorem]
\CreateTheorem{notation}[theorem]
\CreateTheorem{example}[theorem]
\CreateTheorem{problem}[theorem]
\CreateTheorem{question}[theorem]
\CreateTheorem{exercise}[theorem]
\fi
\CreateTheorem*{definition*}
\CreateTheorem*{axiom*}
\CreateTheorem*{assumption*}
\CreateTheorem*{convention*}
\CreateTheorem*{hypothesis*}
\CreateTheorem*{notation*}
\CreateTheorem*{example*}
\CreateTheorem*{problem*}
\CreateTheorem*{question*}
\CreateTheorem*{exercise*}
%
\if@minimalist@nothmnum
\CreateTheorem*{remark}
\CreateTheorem*{observation}
\else
\CreateTheorem@thmnum{remark}
\CreateTheorem@thmnum{observation}
\fi
\CreateTheorem*{remark*}
\CreateTheorem*{observation*}
\fi
}

\ifbool{IsBook}{}{%
%# 针对文章的一些设定
%# Some configurations for articles

%%================================
%% Title block style
%%================================
\if@minimalist@classical
\renewcommand{\@maketitle}{%
    \LocallyStopLineNumbers%
    \begin{center}%
        {\conditionalbfseries\minimalist@titlefontshape\large\@title}\\\bigskip%
        \color{maintext!80!paper}%
        {\small\scshape\@author}\\[2pt]%
        {\small\@date}%
    \end{center}%
    \ifx\@date\@empty\bigskip\fi%
    \bigskip\par%
    \ResumeLineNumbers%
}
\else
\renewcommand{\@maketitle}{%
    \LocallyStopLineNumbers%
    \noindent%
    {\textcolor{maintext!27!paper}{\rule{\textwidth}{0.75pt}}}%
    % \vspace{-\parskip}%
    \vspace{-.5\baselineskip}%
    \begin{flushright}%
        {\conditionalbfseries\@title}\\\medskip%
        \color{maintext!80!paper}%
        {\small\scshape\@author}%
        \par\vspace{-\parskip}\vspace{2pt}%
        {\small\@date}%
    \end{flushright}%
    % \vspace{-\parskip}%
    \vspace{-.5\baselineskip}%
    \ifx\@date\@empty%
        \vspace{\baselineskip}%
        \vspace{1.2\parskip}%
    \else%
        \vspace{-.5\baselineskip}%
    \fi%
    {\textcolor{maintext!27!paper}{\rule{\textwidth}{0.75pt}}\par}%
    \ResumeLineNumbers%
}
\fi
\apptocmd{\maketitle}{\thispagestyle{fancy}}{}{\FAIL}

%%================================
%% Abstract style
%%================================
\if@minimalist@classical
\renewenvironment{abstract}{%
    \LocallyStopLineNumbers%
    \vspace{-\baselineskip}%
    \begin{center}%
        \textsc{\conditionalbfseries\small\abstractname}\\%
        \vspace{-.3\baselineskip}%
        \begin{minipage}[t]{.833\textwidth}%
            \vspace{0pt}%
            \color{maintext!80!paper}%
            \footnotesize%
            \parindent=2em
}{%
        \end{minipage}%
    \end{center}%
    \medskip%
    \ResumeLineNumbers%
}
\else
\renewenvironment{abstract}{%
    \LocallyStopLineNumbers%
    \begin{flushright}%
        \textsc{\conditionalbfseries\small\abstractname}\par%
        \vspace{-\parskip}%
        \vspace{-.25\baselineskip}%
        \begin{minipage}[t]{.833\textwidth}%
            \vspace{0pt}%
            \color{maintext!80!paper}%
            \footnotesize%
            \parindent=2em
}{%
        \end{minipage}%
    \end{flushright}%
    \bigskip%
    \ResumeLineNumbers%
}
\fi

%%================================
%% Keyword environment
%%================================
\def\keywordnameEN{\textsc{Keywords}}
\def\keywordnameFR{\textsc{Mots clés}}
\def\keywordnameDE{\textsc{Schlüsselwörter}}
\def\keywordnameCN{关键词}
\def\keywordnameTC{關鍵詞}
\def\keywordnameJP{キーワード}
\def\keywordnameRU{Ключевые слова}

\if@minimalist@classical
\newenvironment{keyword}{%
    \LocallyStopLineNumbers%
    \vspace{-.75\baselineskip}%
    \begin{center}%
        {\conditionalbfseries\small\csname keywordname\csname\languagename ABBR\endcsname\endcsname}\\%
        \vspace{-.3\baselineskip}%
        \begin{minipage}[t]{.833\textwidth}%
            \vspace{0pt}%
            \color{maintext!80!paper}%
            \footnotesize%
            \parindent=2em
            \begin{center}%
}{%
            \end{center}%
        \end{minipage}%
    \end{center}%
    \medskip%
    \ResumeLineNumbers%
}
\else
\newenvironment{keyword}{%
    \LocallyStopLineNumbers%
    \vspace{-.5\baselineskip}%
    \begin{flushright}%
        {\conditionalbfseries\small\csname keywordname\csname\languagename ABBR\endcsname\endcsname}\par%
        \vspace{-\parskip}%
        \vspace{-.30\baselineskip}%
        \begin{minipage}[t]{.833\textwidth}%
            \vspace{0pt}%
            \color{maintext!80!paper}%
            \footnotesize%
            \parindent=2em
            \raggedleft%
}{%
        \end{minipage}%
    \end{flushright}%
    \bigskip%
    \ResumeLineNumbers%
}
\fi

%%================================
%% Simulate features of amsart
%%================================
%# 引入 ProjLib 中用于模拟 AMS 文档类写法的模块
%# Introduce the module of ProjLib that emulates the writing style in AMS classes
\PassOptionsToPackage{amsfashion}{PJLauthor}
\RequirePackage{PJLauthor}

}
%</minimalist>

\endinput