% \iffalse meta-comment
%
% Copyright (C) 2021-2022 by Jinwen XU
% ------------------------------------
%
% This file may be distributed and/or modified under the conditions of the LaTeX
% Project Public License, either version 1.3c of this license or (at your option)
% any later version. The latest version of this license is in:
%
%    http://www.latex-project.org/lppl.txt
%
% \fi
%
%<*driver>
\ProvidesFile{minimalist.dtx}
%</driver>
\NeedsTeXFormat{LaTeX2e}[2020-10-01]
\RequirePackage{l3keys2e}
%
%<*minimart>
\ProvidesExplClass
  {minimart}
  {2022/03/10} {}
  {A simple and clear article style}

\tl_const:Nn \l__minimalist_base_class_tl { article }
%</minimart>
%
%<*minimbook>
\ProvidesExplClass
  {minimbook}
  {2022/03/10} {}
  {A simple and clear book style}

\tl_const:Nn \l__minimalist_base_class_tl { book }
%</minimbook>
%
%<*einfart>
\ProvidesExplClass
  {einfart}
  {2022/03/10} {}
  {A simple and clear article style}

\tl_const:Nn \l__minimclass_base_class_tl { article }
%</einfart>
%
%<*simplivre>
\ProvidesExplClass
  {simplivre}
  {2022/03/10} {}
  {A simple and clear book style}

\tl_const:Nn \l__minimalist_base_class_tl { book }
%</simplivre>
%
%<*minimalist>
\ProvidesExplPackage
  {minimalist}
  {2022/03/10} {}
  {A simple and clear style for articles and books}
%</minimalist>

%<*class>
\keys_define:nn { minimclass }
  {
    , draft             .bool_set:N         = \l__minimclass_fast_bool
    , draft             .initial:n          = { false }
    , fast              .bool_set:N         = \l__minimclass_fast_bool

    , classical         .bool_set:N         = \l__minimclass_classical_bool
    , classical         .initial:n          = { false }

    , use indent        .bool_set:N         = \l__minimclass_useindent_bool
    , use indent        .initial:n          = { true }
    , use~indent        .bool_set:N         = \l__minimclass_useindent_bool
    , use-indent        .bool_set:N         = \l__minimclass_useindent_bool

    , a4paper           .bool_set:N         = \l__minimclass_a_four_paper_bool
    , a4paper           .initial:n          = { false }
    , b5paper           .bool_set:N         = \l__minimclass_b_five_paper_bool
    , b5paper           .initial:n          = { false }

    , unknown           .code:n             = {
                                                \PassOptionsToPackage { \CurrentOption } { minimalist }
                                              }
  }
\ProcessKeysOptions { minimclass }

\LoadClass{\l__minimclass_base_class_tl}

\bool_if:NT \l__minimclass_classical_bool
  {
    \bool_set_false:N \l__minimclass_useindent_bool
  }

%%================================
%%  Page layout
%%================================
\RequirePackage { silence }
\WarningFilter { geometry } { Over-specification }

\PassOptionsToPackage { heightrounded } { geometry }
\RequirePackage { geometry }

\ExplSyntaxOff
\geometry
  {
    papersize = { 7in, 10in },
    % total = { 40em, 60em },
    total = { 5.535in, 8.300in },
    hmarginratio = 1:1,
    vmarginratio = 1:1,
    footnotesep = 2em plus 2pt minus 2pt,
  }
\ExplSyntaxOn

\bool_if:NT \l__minimclass_b_five_paper_bool
  {
    \ExplSyntaxOff
    \geometry
      {
        b5paper,
        % total = { 40em, 59em },
        total = { 5.535in, 8.160in },
        hmarginratio = 1:1,
        vmarginratio = 1:1,
        footnotesep = 2em plus 2pt minus 2pt,
      }
    \ExplSyntaxOn
  }

\bool_if:NT \l__minimclass_a_four_paper_bool
  {
    \ExplSyntaxOff
    \geometry
      {
        a4paper,
        % total = { 47em, 70em },
        total = { 6.500in, 9.685in },
        hmarginratio = 1:1,
        vmarginratio = 1:1,
        footnotesep = 2em plus 2pt minus 2pt,
      }
    \ExplSyntaxOn
  }

\bool_if:NT \l__minimclass_fast_bool
  {
    \PassOptionsToPackage { fast } { minimalist }
    \RequirePackage { draftwatermark }
    \DraftwatermarkOptions { text = { \normalfont DRAFT }, color = paper!95!-paper }
  }

\bool_if:NTF \l__minimclass_useindent_bool
  {
    \RequirePackage { indentfirst }
  }
  {
    \hook_gput_code:nnn { begindocument/before } { minimclass }
      {
        \RequirePackage { parskip }
      }
  }

\RequirePackage { minimalist }

%%================================
%%  Fonts
%%================================
\WarningFilter { latexfont } { Font~shape }
\WarningFilter { latexfont } { Some~font  }

\RequirePackage { projlib-font }

%<*minimart|minimbook>
\bool_if:NF \g_projlib_font_already_set_bool
  {
    \RequirePackage { mathpazo }
    \RequirePackage { newpxtext }
    \RequirePackage { amssymb }
  }
%</minimart|minimbook>
%
%<*einfart|simplivre>
\bool_if:NF \g_projlib_font_already_set_bool
  {
    \bool_if:NT \l__minimclass_fast_bool
      {
        \RequirePackage { mathpazo }
      }

    \PassOptionsToPackage { no-math,quiet } { fontspec }
    \RequirePackage { fontspec }

    \IfFileExists { minimalist-font.latin.tex }
      {
        \input { minimalist-font.latin.tex }
      }
      {
        \IfFontExistsTF { Palatino~Linotype }
          {
            \bool_if:NTF \l__projlib_font_useosf_bool
              {
                \setmainfont { Palatino~Linotype } [ Numbers = OldStyle ]
              }
              {
                \setmainfont { Palatino~Linotype }
              }
          }
          {
            \bool_if:NTF \l__projlib_font_useosf_bool
              {
                \setmainfont { TeXGyrePagellaX-Regular.otf }
                  [
                    BoldFont       = TeXGyrePagellaX-Bold.otf ,
                    ItalicFont     = TeXGyrePagellaX-Italic.otf ,
                    BoldItalicFont = TeXGyrePagellaX-BoldItalic.otf,
                    Numbers        = OldStyle ,
                  ]
              }
              {
                \setmainfont { TeXGyrePagellaX-Regular.otf }
                  [
                    BoldFont       = TeXGyrePagellaX-Bold.otf ,
                    ItalicFont     = TeXGyrePagellaX-Italic.otf ,
                    BoldItalicFont = TeXGyrePagellaX-BoldItalic.otf,
                  ]
              }
          }
        \setsansfont { SourceSansPro-Regular.otf }
          [
            Scale          = MatchLowercase ,
            BoldFont       = SourceSansPro-Bold.otf ,
            ItalicFont     = SourceSansPro-RegularIt.otf ,
            BoldItalicFont = SourceSansPro-BoldIt.otf ,
          ]
        \setmonofont { NewCMMono10-Regular.otf }
          [
            Scale          = 1.05 ,
            BoldFont       = NewCMMono10-Bold.otf ,
            ItalicFont     = NewCMMono10-Italic.otf ,
            BoldItalicFont = NewCMMono10-BoldOblique.otf ,
          ]

        \projlib_language_set_linespacing_latin:n { \setstretch { 1.07 } }
      }

    \PassOptionsToPackage { fontset = none, scheme = plain } { ctex }
    \RequirePackage { ctex }

    \IfFileExists { minimalist-font.cjk.tex }
      {
        \input { minimalist-font.cjk.tex }
      }
      {
        \IfFontExistsTF{SourceHanSerifSC-Regular}
          {
            \setCJKmainfont { SourceHanSerifSC-Regular }
              [
                BoldFont       = SourceHanSerifSC-Bold ,
                ItalicFont     = * ,
                BoldItalicFont = SourceHanSerifSC-Bold ,
              ]
          }
          {
            \setCJKmainfont { FandolSong-Regular.otf }
              [
                BoldFont       = FandolSong-Bold.otf ,
                ItalicFont     = FandolKai-Regular.otf ,
                BoldItalicFont = FandolKai-Regular.otf ,
                BoldItalicFeatures = { FakeBold = 4 } ,
              ]
          }

        \IfFontExistsTF { SourceHanSansSC-Regular }
          {
            \setCJKsansfont{SourceHanSansSC-Regular}
              [
                BoldFont       = SourceHanSansSC-Bold ,
                ItalicFont     = * ,
                BoldItalicFont = SourceHanSansSC-Bold ,
              ]
          }
          {
            \setCJKsansfont { FandolHei-Regular.otf }
              [
                BoldFont       = FandolHei-Bold.otf ,
                ItalicFont     = * ,
                BoldItalicFont = FandolHei-Bold.otf ,
              ]
          }

        \IfFontExistsTF { SourceHanMonoSC-Regular }
          {
            \setCJKmonofont { SourceHanMonoSC-Regular }
              [
                BoldFont       = SourceHanMonoSC-Medium ,
                ItalicFont     = * ,
                BoldItalicFont = SourceHanMonoSC-Medium ,
              ]
          }
          {
            \setCJKmonofont { FandolFang-Regular.otf }
              [
                BoldFont       = * ,
                BoldFeatures   = { FakeBold = 4 } ,
                ItalicFont     = * ,
                BoldItalicFont = * ,
                BoldItalicFeatures = { FakeBold = 4 } ,
              ]
          }

        \bool_if:NT \g__projlib_language_enabled_schinese_bool
          {
            \IfFontExistsTF { SourceHanSerifSC-Regular }
              {
                \setCJKfamilyfont { SCmain } { SourceHanSerifSC-Regular }
                  [
                    BoldFont       = SourceHanSerifSC-Bold ,
                    ItalicFont     = * ,
                    BoldItalicFont = SourceHanSerifSC-Bold ,
                  ]
              }
              {
                \setCJKfamilyfont { SCmain } { FandolSong-Regular.otf }
                  [
                    BoldFont       = FandolSong-Bold.otf ,
                    ItalicFont     = FandolKai-Regular.otf ,
                    BoldItalicFont = FandolKai-Regular.otf ,
                    BoldItalicFeatures = { FakeBold = 4 } ,
                  ]
              }
            \IfFontExistsTF { SourceHanSansSC-Regular }
              {
                \setCJKfamilyfont { SCsans } { SourceHanSansSC-Regular }
                  [
                    BoldFont       = SourceHanSansSC-Bold ,
                    ItalicFont     = * ,
                    BoldItalicFont = SourceHanSansSC-Bold ,
                  ]
              }
              {
                \setCJKfamilyfont { SCsans } { FandolHei-Regular.otf }
                  [
                    BoldFont       = FandolHei-Bold.otf ,
                    ItalicFont     = * ,
                    BoldItalicFont = FandolHei-Bold.otf ,
                  ]
              }
            \IfFontExistsTF { SourceHanMonoSC-Regular }
              {
                \setCJKfamilyfont { SCmono } { SourceHanMonoSC-Regular }
                  [
                    BoldFont       = SourceHanMonoSC-Medium ,
                    ItalicFont     = * ,
                    BoldItalicFont = SourceHanMonoSC-Medium ,
                  ]
              }
              {
                \setCJKfamilyfont { SCmono } { FandolFang-Regular.otf }
                  [
                    BoldFont       = * ,
                    BoldFeatures   = { FakeBold = 4 } ,
                    ItalicFont     = * ,
                    BoldItalicFont = * ,
                    BoldItalicFeatures = { FakeBold = 4 } ,
                  ]
              }
          }

        \bool_if:NT \g__projlib_language_enabled_tchinese_bool
          {
            \IfFontExistsTF { SourceHanSerifTC-Regular }
              {
                \setCJKfamilyfont { TCmain } { SourceHanSerifTC-Regular }
                  [
                    BoldFont       = SourceHanSerifTC-Bold ,
                    ItalicFont     = * ,
                    BoldItalicFont = SourceHanSerifTC-Bold ,
                  ]
              }
              {
                \setCJKfamilyfont { TCmain } { FandolSong-Regular.otf }
                  [
                    BoldFont       = FandolSong-Bold.otf ,
                    ItalicFont     = FandolKai-Regular.otf ,
                    BoldItalicFont = FandolKai-Regular.otf ,
                    BoldItalicFeatures = { FakeBold = 4 } ,
                  ]
              }
            \IfFontExistsTF { SourceHanSansTC-Regular }
              {
                \setCJKfamilyfont { TCsans } { SourceHanSansTC-Regular }
                  [
                    BoldFont       = SourceHanSansTC-Bold ,
                    ItalicFont     = * ,
                    BoldItalicFont = SourceHanSansTC-Bold ,
                  ]
              }
              {
                \setCJKfamilyfont { TCsans } { FandolHei-Regular.otf }
                  [
                    BoldFont       = FandolHei-Bold.otf ,
                    ItalicFont     = * ,
                    BoldItalicFont = FandolHei-Bold.otf ,
                  ]
              }
            \IfFontExistsTF { SourceHanMonoTC-Regular }
              {
                \setCJKfamilyfont { TCmono } { SourceHanMonoTC-Regular }
                  [
                    BoldFont       = SourceHanMonoTC-Medium ,
                    ItalicFont     = * ,
                    BoldItalicFont = SourceHanMonoTC-Medium ,
                  ]
              }
              {
                \setCJKfamilyfont { TCmono } { FandolFang-Regular.otf }
                  [
                    BoldFont       = * ,
                    BoldFeatures   = { FakeBold = 4 } ,
                    ItalicFont     = * ,
                    BoldItalicFont = * ,
                    BoldItalicFeatures = { FakeBold = 4 } ,
                  ]
              }
          }

        \bool_if:NT \g__projlib_language_enabled_japanese_bool
          {
            \IfFontExistsTF { SourceHanSerif-Regular }
              {
                \setCJKfamilyfont { JPmain } { SourceHanSerif-Regular }
                  [
                    BoldFont       = SourceHanSerif-Bold ,
                    ItalicFont     = * ,
                    BoldItalicFont = SourceHanSerif-Bold ,
                  ]
              }
              {
                \setCJKfamilyfont { JPmain } { FandolSong-Regular.otf }
                  [
                    BoldFont       = FandolSong-Bold.otf ,
                    ItalicFont     = FandolKai-Regular.otf ,
                    BoldItalicFont = FandolKai-Regular.otf ,
                    BoldItalicFeatures = { FakeBold = 4 } ,
                  ]
              }
            \IfFontExistsTF { SourceHanSans-Regular }
              {
                \setCJKfamilyfont { JPsans } { SourceHanSans-Regular }
                  [
                    BoldFont       = SourceHanSans-Bold ,
                    ItalicFont     = * ,
                    BoldItalicFont = SourceHanSans-Bold ,
                  ]
              }
              {
                \setCJKfamilyfont { JPsans } { FandolHei-Regular.otf }
                  [
                    BoldFont       = FandolHei-Bold.otf ,
                    ItalicFont     = * ,
                    BoldItalicFont = FandolHei-Bold.otf ,
                  ]
              }
            \IfFontExistsTF { SourceHanMono-Regular }
              {
                \setCJKfamilyfont { JPmono } { SourceHanMono-Regular }
                  [
                    BoldFont       = SourceHanMono-Medium ,
                    ItalicFont     = * ,
                    BoldItalicFont = SourceHanMono-Medium ,
                  ]
              }
              {
                \setCJKfamilyfont { JPmono } { FandolFang-Regular.otf }
                  [
                    BoldFont       = * ,
                    BoldFeatures   = { FakeBold = 4 } ,
                    ItalicFont     = * ,
                    BoldItalicFont = * ,
                    BoldItalicFeatures = { FakeBold = 4 } ,
                  ]
              }
          }

        \cs_new:Nn \minimclass_cjk_sffamily: {}
        \cs_new:Nn \minimclass_cjk_ttfamily: {}

        \hook_gput_code:nnn { cmd/sffamily/after } { minimclass } { \minimclass_cjk_sffamily: }
        \hook_gput_code:nnn { cmd/ttfamily/after } { minimclass } { \minimclass_cjk_ttfamily: }

        \AddLanguageSetting [schinese]
          {
            \cs_set:Nn \minimclass_cjk_sffamily: { \CJKfamily { SCsans } }
            \cs_set:Nn \minimclass_cjk_ttfamily: { \CJKfamily { SCmono } }
            \CJKfamily { SCmain }
          }
        \AddLanguageSetting [tchinese]
          {
            \cs_set:Nn \minimclass_cjk_sffamily: { \CJKfamily { TCsans } }
            \cs_set:Nn \minimclass_cjk_ttfamily: { \CJKfamily { TCmono } }
            \CJKfamily { TCmain }
          }
        \AddLanguageSetting [japanese]
          {
            \cs_set:Nn \minimclass_cjk_sffamily: { \CJKfamily { JPsans } }
            \cs_set:Nn \minimclass_cjk_ttfamily: { \CJKfamily { JPmono } }
            \CJKfamily { JPmain }
          }

        \tl_gset:Nn \g_minimalist_title_font_common_tl { \minimclass_cjk_sffamily: }
      }

    \bool_if:NTF \l__minimclass_fast_bool
      {
        \RequirePackage { amssymb }
      }
      {
        \PassOptionsToPackage { warnings-off = { mathtools-colon, mathtools-overbracket } } { unicode-math }
        \RequirePackage { unicode-math }
        \unimathsetup { math-style = ISO, partial = upright, nabla = upright }
        \setmathfont { Asana-Math.otf }
        \IfFontExistsTF { Neo~Euler }
          {
            \setmathfont { Neo~Euler }
              [
                range             = { up / { Latin, latin, Greek, greek },
                                      bfup / { Latin, latin, Greek, greek },
                                      cal, bfcal, frak, bffrak,
                                      `(, `), `[, `], `\{, `\}, `:, `=, \ne, \equiv, `/, \backslash,
                                      \in, \notin, \ni, \subset, \supset, \subseteq, \supseteq,
                                      \rightarrow, \leftarrow, \mapsto, \hookrightarrow, \hookleftarrow,
                                      \sum, \prod, \coprod,
                                      \sqrt, \int, \iint, \iiint, \oint
                                    },
                script-features   = {},
                sscript-features  = {},
                Scale             = 1.05,
              ]
          }{}
        \setmathfont { latinmodern-math.otf }
          [
            range = { \leq, \geq, \ll, \gg, \lll, \ggg, \leqslant, \geqslant },
            Scale = 0.95,
          ]
        \setmathfont { texgyrepagella-math.otf }
          [
            range = { up / num, bfup / num }
          ]

        \hook_gput_code:nnn { begindocument } { minimclass }
          {
            \NewCommandCopy \minimclass_backup_mid: \mid
            \RenewDocumentCommand \mid {}
              { \skip_horizontal:n {-.15em} \minimclass_backup_mid: \skip_horizontal:n {-.15em} }
            \NewCommandCopy \minimclass_backup_nmid: \nmid
            \RenewDocumentCommand \nmid {}
              { \skip_horizontal:n {-.15em} \minimclass_backup_nmid: \skip_horizontal:n {-.15em} }

            \RenewDocumentCommand \frac { m m }
              {
                \genfrac{}{}{}{}
                  {
                    \mathchoice
                      { \raisebox{-.15em}{$\displaystyle #1$} }
                      { \raisebox{-.15em}{$\textstyle #1$} }
                      { \raisebox{-.08em}{$\scriptstyle #1$} }
                      { \scriptscriptstyle #1 }
                  }
                  {
                    \mathchoice
                      { \raisebox{.08em}{$\displaystyle #2$} }
                      { \raisebox{.08em}{$\textstyle #2$} }
                      { \raisebox{.05em}{$\scriptstyle #2$} }
                      { \scriptscriptstyle #2 }
                  }
              }
          }

        \RequirePackage { tikz-cd }

        \box_new:N \l__minimclass_xarrows_above_box
        \box_new:N \l__minimclass_xarrows_below_box
        \dim_new:N \l__minimclass_xarrows_length_dim
        \cs_new_protected:Nn \minimclass_xarrows_generic:nnnn
          % #3 = option of \tikz
          % #4 = edge of \draw
          {
            \hbox_set:Nn \l__minimclass_xarrows_below_box { \ensuremath { \scriptstyle #1 } }
            \hbox_set:Nn \l__minimclass_xarrows_above_box { \ensuremath { \scriptstyle #2 } }
            \dim_set:Nn \l__minimclass_xarrows_length_dim
              { \dim_eval:n { \dim_max:nn { \box_wd:N \l__minimclass_xarrows_below_box } { \box_wd:N \l__minimclass_xarrows_above_box } + .8em } }
            \mathrel
              {
                \tikz [ #3, baseline = -.55ex, every~node/.style = { inner~sep = 0pt } ]
                  \draw (0,0) #4
                    node [ below = 3pt ] { \box_use:N \l__minimclass_xarrows_below_box }
                    node [ above = 2pt ] { \box_use:N \l__minimclass_xarrows_above_box }
                    ( \l__minimclass_xarrows_length_dim ,0) ;
              }
          }

        \RenewDocumentCommand \xrightarrow { O{} m }
          {
            \minimclass_xarrows_generic:nnnn { #1 } { #2 } { -> } { -- }
          }
        \RenewDocumentCommand \xleftarrow { O{} m }
          {
            \minimclass_xarrows_generic:nnnn { #1 } { #2 } { <- } { -- }
          }
        \RenewDocumentCommand \xleftrightarrow { O{} m }
          {
            \minimclass_xarrows_generic:nnnn { #1 } { #2 } { <-> } { -- }
          }
        \RenewDocumentCommand \xhookrightarrow { O{} m }
          {
            \minimclass_xarrows_generic:nnnn { #1 } { #2 } {} { edge [ commutative~diagrams/hookrightarrow ] }
          }
        \RenewDocumentCommand \xhookleftarrow { O{} m }
          {
            \minimclass_xarrows_generic:nnnn { #1 } { #2 } {} { edge [ commutative~diagrams/hookleftarrow ] }
          }
        \RenewDocumentCommand \xmapsto { O{} m }
          {
            \minimclass_xarrows_generic:nnnn { #1 } { #2 } {} { edge [ commutative~diagrams/mapsto ] }
          }
        \NewDocumentCommand \xlongequal { O{} m }
          {
            \minimclass_xarrows_generic:nnnn { #1 } { #2 } {} { edge [ commutative~diagrams/equal ] }
          }
        \hook_gput_code:nnn { begindocument/end } { minimclass }
          {
            \RenewDocumentCommand \twoheadrightarrow {}
              {
                \minimclass_xarrows_generic:nnnn { \,\, } {} {} { edge [ commutative~diagrams/twoheadrightarrow ] }
              }
            \RenewDocumentCommand \twoheadleftarrow {}
              {
                \minimclass_xarrows_generic:nnnn { \,\, } {} {} { edge [ commutative~diagrams/twoheadleftarrow ] }
              }
          }
      }
  }
%</einfart|simplivre>

\PassOptionsToPackage { all } { nowidow }
\RequirePackage { nowidow }
\RequirePackage { embrac }

%%================================
%%  Graphics
%%================================
\RequirePackage { graphicx }
\graphicspath { { images/ } }
\RequirePackage { wrapfig }
\RequirePackage { float }
\RequirePackage { caption }
\captionsetup { font = small }
%</class>
%
%
%<*minimalist>
\keys_define:nn { minimalist }
  {
    , draft             .bool_set:N         = \l__minimalist_fast_bool
    , draft             .initial:n          = { false }
    , fast              .bool_set:N         = \l__minimalist_fast_bool

    , classical         .bool_set:N         = \l__minimalist_classical_bool
    , classical         .initial:n          = { false }

    , use-boldface      .bool_set:N         = \l__minimalist_use_boldface_bool
    , use-boldface      .initial:n          = { false }
    , use~boldface      .bool_set:N         = \l__minimalist_use_boldface_bool
    , use boldface      .bool_set:N         = \l__minimalist_use_boldface_bool
    , usebf             .bool_set:N         = \l__minimalist_use_boldface_bool
    , allow-boldface    .bool_set:N         = \l__minimalist_use_boldface_bool
    , allow~boldface    .bool_set:N         = \l__minimalist_use_boldface_bool
    , allow boldface    .bool_set:N         = \l__minimalist_use_boldface_bool
    , allowbf           .bool_set:N         = \l__minimalist_use_boldface_bool
    , runin             .bool_set:N         = \l__minimalist_runin_bool
    , runin             .initial:n          = { false }

    , theorem-in-new-line .bool_set:N       = \l__minimalist_use_theorem_in_new_line_bool
    , theorem-in-new-line .initial:n        = { false }
    , theorem~in~new~line .bool_set:N       = \l__minimalist_use_theorem_in_new_line_bool
    , theorem in new line .bool_set:N       = \l__minimalist_use_theorem_in_new_line_bool

    , unknown           .code:n             = {
                                                \PassOptionsToPackage { \CurrentOption } { projlib-language }
                                                \PassOptionsToPackage { \CurrentOption } { projlib-author }
                                                \PassOptionsToPackage { \CurrentOption } { projlib-datetime }
                                                \PassOptionsToPackage { \CurrentOption } { projlib-draft }
                                                \PassOptionsToPackage { \CurrentOption } { projlib-font }
                                                \PassOptionsToPackage { \CurrentOption } { projlib-logo }
                                                \PassOptionsToPackage { \CurrentOption } { projlib-math }
                                                \PassOptionsToPackage { \CurrentOption } { projlib-paper }
                                                \PassOptionsToPackage { \CurrentOption } { projlib-theorem }
                                              }
  }
\ProcessKeysOptions { minimalist }

\bool_new:N \l__minimalist_is_book_bool
\cs_if_exist:cTF { c@chapter }
  {
    \bool_set_true:N \l__minimalist_is_book_bool
  }
  {
    \bool_set_false:N \l__minimalist_is_book_bool
  }

%%================================
%%  Paper configuration
%%================================
\RequirePackage { projlib-paper }

%%================================
%%  Multi-language support
%%================================
\RequirePackage { projlib-language }

%%================================
%%  Title fonts
%%================================
\RequirePackage { anyfontsize }

\bool_if:NTF \l__minimalist_use_boldface_bool
  {
    \cs_new:Nn \minimalist_bfseries: { \bfseries \colorlet{minimalist-temp-color}{.} \color{minimalist-temp-color!70!paper} }
  }
  {
    \cs_new:Nn \minimalist_bfseries: {}
  }

\tl_new:N \g_minimalist_title_font_common_tl

\tl_new:N \g_minimalist_title_font_part_tl
\tl_new:N \g_minimalist_title_font_chapter_tl
\tl_new:N \g_minimalist_title_font_section_tl
\tl_new:N \g_minimalist_title_font_subsection_tl
\tl_new:N \g_minimalist_title_font_subsubsection_tl

\bool_if:NTF \l__minimalist_classical_bool
  {
    \tl_gset:Nn \g_minimalist_title_font_part_tl          { \minimalist_bfseries: \g_minimalist_title_font_common_tl }
    \tl_gset:Nn \g_minimalist_title_font_chapter_tl       { \minimalist_bfseries: \g_minimalist_title_font_common_tl }
    \tl_gset:Nn \g_minimalist_title_font_section_tl       { \minimalist_bfseries: }
    \tl_gset:Nn \g_minimalist_title_font_subsection_tl    { \minimalist_bfseries: }
    \tl_gset:Nn \g_minimalist_title_font_subsubsection_tl { \minimalist_bfseries: \itshape }
  }
  {
    \tl_gset:Nn \g_minimalist_title_font_part_tl          { \minimalist_bfseries: \g_minimalist_title_font_common_tl }
    \tl_gset:Nn \g_minimalist_title_font_chapter_tl       { \minimalist_bfseries: \g_minimalist_title_font_common_tl }
    \tl_gset:Nn \g_minimalist_title_font_section_tl       { \minimalist_bfseries: \g_minimalist_title_font_common_tl \scshape }
    \tl_gset:Nn \g_minimalist_title_font_subsection_tl    { \minimalist_bfseries: \g_minimalist_title_font_common_tl \scshape }
    \tl_gset:Nn \g_minimalist_title_font_subsubsection_tl { \minimalist_bfseries: \g_minimalist_title_font_common_tl }
  }

%%================================
%%  Footer
%%================================
\RequirePackage { geometry }
\RequirePackage { fancyhdr }
\RequirePackage { extramarks }

\hook_gput_code:nnn { begindocument/before } { minimalist }
  {
    \fancyhfoffset { 0pt }
  }

\fancypagestyle { fancy }
  {
    \fancyhf{}
    \if@twoside
      \fancyfoot[RO]{\small\textcolor{main-text!30!paper}{\lastrightmark}
        \nobreakspace\nobreakspace\rlap{\textcolor{main-text!27!paper}{$|$}\nobreakspace\nobreakspace\thepage}}
      \fancyfoot[LE]{\small\leavevmode\llap{\thepage
        \nobreakspace\nobreakspace\textcolor{main-text!27!paper}{$|$}}
        \nobreakspace\nobreakspace\textcolor{main-text!30!paper}{\lastleftmark}}
    \else
      \fancyfoot[R]{\small\textcolor{main-text!30!paper}{\lastrightmark}
        \nobreakspace\nobreakspace\rlap{\textcolor{main-text!27!paper}{$|$}\nobreakspace\nobreakspace\thepage}}
    \fi
    \renewcommand{\headrulewidth}{0pt}
  }
\pagestyle{fancy}

\fancypagestyle { plain }
  {
    \fancyhf{}
    \if@twoside
      \fancyfoot[RO]{\small
        \nobreakspace\rlap{\textcolor{main-text!27!paper}{$|$}\nobreakspace\nobreakspace\thepage}}
      \fancyfoot[LE]{\small\leavevmode\llap{\thepage
        \nobreakspace\nobreakspace\textcolor{main-text!27!paper}{$|$}}}
    \else
      \fancyfoot[R]{\small
        \nobreakspace\rlap{\textcolor{main-text!27!paper}{$|$}\nobreakspace\nobreakspace\thepage}}
    \fi
    \renewcommand{\headrulewidth}{0pt}
  }

\bool_if:NTF \l__minimalist_is_book_bool
  {
    \bool_if:NTF \l__minimalist_fast_bool
      {
        \newcommand{\drawHelpLine}{}
      }
      {
        \RequirePackage { tikz }
        \ExplSyntaxOff
        \usetikzlibrary{calc,shadings}
        \ExplSyntaxOn
        \RequirePackage { tikzpagenodes } % For `current page text area`
        \newcommand{\drawHelpLine}{
            \begin{tikzpicture}[remember~picture,overlay]
                \foreach\i in {0,1,...,5}{
                    \fill[opacity=0.12-0.02*\i]
                        ($(current~page~text~area.north~east)
                            +(-\i*0.5em-.025em,-10pt+\i*1.1pt)$)
                        rectangle ($(current~page~text~area.south~east)
                            +(-\i*0.5em+.025em,10pt-\i*1.1pt)$);
                    \shade[top~color=paper,bottom~color=main-text,opacity=0.12-0.02*\i]
                        ($(current~page~text~area.north~east)
                            +(-\i*0.5em-.025em,2pt)$)
                        rectangle ($(current~page~text~area.north~east)
                            +(-\i*0.5em+.025em,-10pt+\i*1.1pt)$);
                    \shade[top~color=main-text,bottom~color=paper,opacity=0.12-0.02*\i]
                        ($(current~page~text~area.south~east)
                            +(-\i*0.5em-.025em,-2pt)$)
                        rectangle ($(current~page~text~area.south~east)
                            +(-\i*0.5em+.025em,10pt-\i*1.1pt)$);
                }
            \end{tikzpicture}
        }
      }
    \fancypagestyle{part}{
        \fancyhf{}
        \renewcommand{\headrulewidth}{0pt}
        \fancyhead[C]{\drawHelpLine}
    }
    \addtolength{\headheight}{20pt}
    \addtolength{\topmargin}{-20pt}
    \if@twoside
        \renewcommand{\chaptermark}[1]{\markboth{\textsc{#1}}{}}
    \else
        \renewcommand{\chaptermark}[1]{\markboth{\textsc{#1}}{\textsc{#1}}}
    \fi
    \renewcommand*{\sectionmark}[1]{
        \markright{\sec@decochar\nobreakspace\arabic{section}\nobreakspace\sec@decochar\nobreakspace\nobreakspace\nobreakspace#1}}
  }
  {
    \if@twoside
        \renewcommand*{\sectionmark}[1]{\markboth{\textsc{#1}}{}}
    \else
        \renewcommand*{\sectionmark}[1]{\markboth{\textsc{#1}}{\textsc{#1}}}
    \fi
  }

%%================================
%%  Line numbers
%%================================
\PassOptionsToPackage { pagewise,mathlines } { lineno }
\RequirePackage { linenoamsmath }
\renewcommand{\linenumberfont}{\ttfamily\color{main-text!7!paper}\footnotesize}
\setlength{\linenumbersep}{1em}

\newif\ifLNturnsON
\def\LocallyStopLineNumbers{\LNturnsONfalse
    \ifLineNumbers\LNturnsONtrue\fi\nolinenumbers}
\def\ResumeLineNumbers{\ifLNturnsON\linenumbers\fi}

\hook_gput_code:nnn { cmd/tableofcontents/before } { minimalist } { \LocallyStopLineNumbers }
\hook_gput_code:nnn { cmd/tableofcontents/after } { minimalist } { \ResumeLineNumbers }
\hook_gput_code:nnn { env/bibliography/before } { minimalist } { \LocallyStopLineNumbers }
\hook_gput_code:nnn { env/bibliography/after } { minimalist } { \ResumeLineNumbers }

%%================================
%%  Title format
%%================================
\RequirePackage [ explicit, newparttoc ] { titlesec }
\PassOptionsToPackage { normalem } { ulem }
\RequirePackage { ulem }

\newcommand{\partstring}{\MakeUppercase{{\partname\nobreakspace\protect\thepart}}}

\AddLanguageSetting
  {
    \renewcommand{\partstring}{\MakeUppercase{{\partname\nobreakspace\protect\thepart}}}
  }
\AddLanguageSetting [ schinese ]
  {
    \renewcommand{\partstring}{第 \nobreakspace\thepart\nobreakspace 部分}
  }
\AddLanguageSetting [ tchinese ]
  {
    \renewcommand{\partstring}{第 \nobreakspace\thepart\nobreakspace 部分}
  }
\AddLanguageSetting [ japanese ]
  {
    \renewcommand{\partstring}{第 \nobreakspace\thepart\nobreakspace 部}
  }

\bool_if:NTF \l__minimalist_is_book_bool
  {
    \setcounter{secnumdepth}{3}

    %% Part
    \titleclass{\part}{top} % make part like a chapter
    \titleformat{\part}[display]
      {\thispagestyle{part}
      \LocallyStopLineNumbers
      \g_minimalist_title_font_part_tl\filleft}
      {\partstring}
      {1em}
      {\fontsize{20}{0}\selectfont\MakeUppercase{#1}}
      [\ResumeLineNumbers]
    \titleformat{name=\part,numberless}[display]
      {\thispagestyle{part}
      \LocallyStopLineNumbers
        \phantomsection\addcontentsline{toc}{part}{#1}
      \g_minimalist_title_font_part_tl\filleft}
      {\phantom{\MakeUppercase{\partname}}}
      {1em}
      {\fontsize{20}{0}\selectfont\MakeUppercase{#1}}
      [\ResumeLineNumbers]
    \titlespacing*{\part}{0pt}{5em}{6em}
    %% Text after part
    \newcommand{\parttext}[1]{
    \vfill
    \LocallyStopLineNumbers
    \begin{flushright}
      \begin{minipage}{0.833\textwidth}
        \color{main-text!80!paper}\raggedleft#1
      \end{minipage}
    \end{flushright}
    \ResumeLineNumbers
    \vfill\vfill
    \cleardoublepage
    }

    %% Chapter
    \titleformat{\chapter}
      {\thispagestyle{fancy}
      \LocallyStopLineNumbers
      \color{main-text!80!paper}\g_minimalist_title_font_chapter_tl\fontsize{16}{0}\selectfont}{}{0em}
      {\rlap{\hspace*{-.5em}{\color{main-text!12!paper}
        \fontsize{80}{0}\selectfont\raisebox{-7pt}{\thechapter}}}#1}
      [\ResumeLineNumbers]
    \titleformat{name=\chapter,numberless}
      {\thispagestyle{fancy}
      \LocallyStopLineNumbers
        \phantomsection\addcontentsline{toc}{chapter}{#1}
      \color{main-text!80!paper}\g_minimalist_title_font_chapter_tl\fontsize{16}{0}\selectfont}{}{0em}
      {\rlap{\hspace*{-.5em}{\color{main-text!12!paper}
        \fontsize{80}{0}\selectfont\normalfont\raisebox{-7pt}{*}}}#1}
      [\ResumeLineNumbers]
  }
  {
    %% Part
    \titleformat{\part}[display]
      {\LocallyStopLineNumbers
      \g_minimalist_title_font_part_tl\filleft}
      {\partstring}
      {.3em}
      {\fontsize{16}{0}\selectfont\MakeUppercase{#1}}
      [\ResumeLineNumbers]
    \titleformat{name=\part,numberless}[display]
      {\LocallyStopLineNumbers
        \phantomsection\addcontentsline{toc}{part}{#1}
      \g_minimalist_title_font_part_tl\filleft}
      {\phantom{\MakeUppercase{\partname}}}
      {.3em}
      {\fontsize{16}{0}\selectfont\MakeUppercase{#1}}
      [\ResumeLineNumbers]
    %% Text after part
    \newcommand{\parttext}[1]{
      \LocallyStopLineNumbers
      \begin{flushright}
        \begin{minipage}{0.833\textwidth}
          \color{main-text!80!paper}\raggedleft#1
        \end{minipage}
      \end{flushright}
      \ResumeLineNumbers
    }
  }

%% Section
\bool_if:NTF \l__minimalist_classical_bool
  {
    \renewcommand\thesection{\arabic{section}}
    \newcommand\seculine{\bgroup\markoverwith{\color{main-text!27!paper}
        \rule[-0.9ex]{2pt}{.6pt}\hspace{-2pt}\rule[-1.2ex]{2pt}{.6pt}}\ULon}
    \newcommand\sec@decochar{\raisebox{.03em}{\normalfont/}}
    \titleformat{\section}
      {\LocallyStopLineNumbers
      \g_minimalist_title_font_section_tl\centering}{}{0em}
      {{\small\textcolor{main-text!27!paper}{\footnotesize\sec@decochar}
        \,\,\textcolor{main-text!90!paper}{\minimalist_bfseries:\arabic{section}}
        \,\,\textcolor{main-text!27!paper}{\footnotesize\sec@decochar}}\\
        \seculine{#1}}
      [\ResumeLineNumbers]
    \titleformat{name=\section,numberless}
      {\LocallyStopLineNumbers
        \phantomsection\addcontentsline{toc}{section}{#1}
      \g_minimalist_title_font_section_tl\centering}{}{0em}
      {\seculine{#1}}
      [\ResumeLineNumbers]
  }
  {
    \newcommand\sec@decochar{}
    \titleformat{\section}
      {\LocallyStopLineNumbers
      \g_minimalist_title_font_section_tl\centering}
      {\textcolor{main-text!39!paper}{\usefont{U}{zeur}{b}{n}\arabic{section}}}{.75em}
      {#1}
      [\ResumeLineNumbers]
  }

%% Subsection
\bool_if:NTF \l__minimalist_classical_bool
  {
    \renewcommand\thesubsection{
      \ifnum\c@section=0\else\arabic{section}.\fi\arabic{subsection}}
    \newcommand\subseculine{\bgroup\markoverwith{\color{main-text!27!paper}
      \rule[-1ex]{2pt}{.75pt}}\ULon}
    \titleformat{\subsection}
      {\LocallyStopLineNumbers
      \g_minimalist_title_font_subsection_tl}{}{0em}
      {\subseculine{\thesubsection\nobreakspace\textcolor{main-text!27!paper}{$|$}\nobreakspace #1}}
      [\ResumeLineNumbers]
    \titleformat{name=\subsection,numberless}
      {\LocallyStopLineNumbers
      \g_minimalist_title_font_subsection_tl}{}{0em}
      {\subseculine{#1}}
      [\ResumeLineNumbers]
  }
  {
    \titleformat{\subsection}
      {\LocallyStopLineNumbers
      \g_minimalist_title_font_subsection_tl}
      {\textcolor{main-text!39!paper}{\usefont{U}{zeur}{b}{n}\arabic{section}$.$\arabic{subsection}}}{.5em}
      {#1}
      [\ResumeLineNumbers]
  }

%% Subsubsection
\bool_if:NTF \l__minimalist_classical_bool
  {
    \bool_if:NTF \l__minimalist_runin_bool
      {
        \titleformat{\subsubsection}[runin]
          {\color{main-text!70!paper}\g_minimalist_title_font_subsubsection_tl}
          {\scalebox{0.9}{\thesubsubsection}}{.33em}
          {#1.}[\hspace*{.3em}]
      }
      {
        \titleformat{\subsubsection}
          {\LocallyStopLineNumbers
          \color{main-text!70!paper}\g_minimalist_title_font_subsubsection_tl}
          {\scalebox{0.9}{\thesubsubsection}}{.33em}
          {#1}
          [\ResumeLineNumbers]
      }
  }
  {
    \bool_if:NTF \l__minimalist_runin_bool
      {
        \titleformat{\subsubsection}[runin]
          {\g_minimalist_title_font_subsubsection_tl}
          { \textcolor{main-text!39!paper}{\usefont{U}{zeur}{b}{n}\arabic{section}$.$\arabic{subsection}$.$\arabic{subsubsection}} }{.5em}
          {#1.}[\hspace*{.3em}]
      }
      {
        \titleformat{\subsubsection}
          {\LocallyStopLineNumbers
          \g_minimalist_title_font_subsubsection_tl}
          { \textcolor{main-text!39!paper}{\usefont{U}{zeur}{b}{n}\arabic{section}$.$\arabic{subsection}$.$\arabic{subsubsection}} }{.5em}
          {#1}
          [\ResumeLineNumbers]
      }
  }

%% Paragraph
\titleformat{\paragraph}[runin]
  {\scshape}{\theparagraph}{1em}{#1}

\titlespacing{\section}{0pt}{\baselineskip}{.6\baselineskip}
\titlespacing{\subsection}{0pt}{.75\baselineskip}{.4\baselineskip}
\titlespacing{\subsubsection}{0pt}{.5\baselineskip}{.3\baselineskip}

%%================================
%%  ToC format
%%================================
\RequirePackage { titletoc }
\titlecontents{part}
  [0em]
  {\addvspace{1.5pc}\filcenter\normalfont}
  {\thecontentslabel\nopagebreak\\\nopagebreak\uppercase}
  {}
  {} % without page number
  [\addvspace{.5pc}]
\bool_if:NTF \l__minimalist_is_book_bool
  {
    \titlecontents{chapter}
      [2em] % i.e., 0em (part) + 2em
      {\addvspace{.5pc}\normalfont}
      {\contentslabel{2em}}
      {\hspace*{-2em}}
      {\titlerule*[1em]{\textcolor{main-text!15!paper}{.}}\contentspage}
    \titlecontents{section}
      [4em] % i.e., 2em (chapter) + 2em
      {\normalfont}
      {\contentslabel[\textcolor{main-text!27!paper}{\small\sec@decochar}\,\textcolor{main-text!90!paper}{\thecontentslabel}\,\textcolor{main-text!27!paper}{\small\sec@decochar}]{2em}}
      {\hspace*{-2em}}
      {\titlerule*[1em]{\textcolor{main-text!15!paper}{.}}\contentspage}
    \titlecontents{subsection}
      [6.5em] % i.e., 4em (section) + 2.5em
      {\normalfont}
      {\contentslabel{2.25em}}
      {\hspace*{-2.25em}}
      {\titlerule*[1em]{\textcolor{main-text!15!paper}{.}}\contentspage}
    \titlecontents{subsubsection}
      [8.5em] % i.e., 6.5em (subsection) + 3em
      {\normalfont}
      {\contentslabel{2.75em}}
      {\hspace*{-2.75em}}
      {\titlerule*[1em]{\textcolor{main-text!15!paper}{.}}\contentspage}
  }
  {
    \titlecontents{section}
      [2em] % i.e., 0em (part) + 2em
      {\normalfont}
      {\contentslabel{1.75em}}
      {\hspace*{-1.75em}}
      {\titlerule*[1em]{\textcolor{main-text!15!paper}{.}}\contentspage}
    \titlecontents{subsection}
      [4.5em] % i.e., 2em (section) + 2.5em
      {\normalfont}
      {\contentslabel{2.25em}}
      {\hspace*{-2.25em}}
      {\titlerule*[1em]{\textcolor{main-text!15!paper}{.}}\contentspage}
    \titlecontents{subsubsection}
      [7.5em] % i.e., 4.5em (subsection) + 3em
      {\normalfont}
      {\contentslabel{2.75em}}
      {\hspace*{-2.75em}}
      {\titlerule*[1em]{\textcolor{main-text!15!paper}{.}}\contentspage}
  }

%%================================
%%  Lists
%%================================
\RequirePackage { enumitem }
\setlist{noitemsep}
% \setlist{topsep = .1\baselineskip, parsep=0pt, itemsep=.05\baselineskip}
\setlist[enumerate]{labelsep=*, leftmargin=*}
\setlist[enumerate,1]{label=\arabic*$)$,
    ref = \arabic*$)$}
\setlist[enumerate,2]{label=\emph{\roman*}$)$,
    ref = \arabic{enumi}.\emph{\roman*}$)$}
\setlist[enumerate,3]{label=\emph{\alph*}$)$,
    ref = \arabic{enumi}.\emph{\roman{enumii}}.\emph{\alph*}$)$}
\setlist[description]{font=\normalfont\minimalist_bfseries:}
\bool_if:NT \l__minimalist_classical_bool
  {
    \newcommand\desculine{\colorlet{currentcolor}{.}\bgroup\markoverwith{\color{currentcolor!55!paper}
      \rule[-.45ex]{2pt}{.75pt}}\ULon}
    \renewcommand{\descriptionlabel}[1]{
      \hspace{\labelsep}\normalfont\desculine{#1}
    }
  }

\setlist[itemize]{leftmargin=*}
\AddLanguageSetting { \setlist[itemize,1]{label=\colorlet{currentcolor}{.}\textcolor{currentcolor!27!paper}{$\bullet$}} }
\AddLanguageSetting [french] { \setlist[itemize,1]{label=\colorlet{currentcolor}{.}\textcolor{currentcolor!55!paper}{---}} }
\setlist[itemize,2]{label=\colorlet{currentcolor}{.}\textcolor{currentcolor!27!paper}{--}}
\setlist[itemize,3]{label=\colorlet{currentcolor}{.}\textcolor{currentcolor!27!paper}{\texttt{*}}}

%%================================
%%  Blank page
%%================================
\newcommand{\blinkpagetext}{This~page~is~intentionally~left~blank}
\renewcommand{\cleardoublepage}{
  \relax
  \clearpage
  \if@twoside\ifodd\c@page\relax\else
  \thispagestyle{empty}
  \hook_gput_next_code:nn { shipout/background }
    {
      \put(0.5\paperwidth,-0.5\paperheight){
      \makebox[0pt]{\large\color{main-text!10!paper}\blinkpagetext}}
    }
  \null\newpage\fi\fi
}

%%================================
%%  Draft mark
%%================================
\RequirePackage { projlib-draft }

%%================================
%%  Theorems
%%================================
\RequirePackage { mathtools }
\RequirePackage { amsthm }

\bool_if:NTF \l__minimalist_classical_bool
  {
    \def\simpleqedsymbol{
      \makebox[1em]{\rlap{\textcolor{main-text!12!paper}{\rule[-0.1em]{.95em}{.95em}}}{\kern.07em\raisebox{.07em}{\textcolor{paper}{\rule[-0.1em]{.81em}{.81em}}}\kern.07em}}}
    \bool_if:NTF \l__minimalist_use_theorem_in_new_line_bool
      {
        \newtheoremstyle{simple}
          {.5\baselineskip}{.5\baselineskip}
          {\normalfont}{}
          {\normalfont}{}
          {\newline}
          { \global\let\qedsymbol\simpleqedsymbol
            {\thmname{#1}\nobreakspace\thmnumber{#2}}
            \thmnote{\hspace{.4em}\textcolor{main-text!27!paper}{$|$}\hspace{.4em}\color{main-text!50!paper}\ensuremath{(\text{#3})}}
            \smallskip
            \pushQED{\qed}
          }
      }
      {
        \newtheoremstyle{simple}
          {.5\baselineskip}{.5\baselineskip}
          {\normalfont}{}
          {\normalfont}{}
          {0pt}
          { \global\let\qedsymbol\simpleqedsymbol
            {\thmname{#1}\nobreakspace\thmnumber{#2}}\hspace{.4em}
            \textcolor{main-text!27!paper}{$|$}\hspace{.4em}
            \color{main-text!50!paper}\thmnote{\ensuremath{(\text{#3})}\nobreakspace\nobreakspace}
            \pushQED{\qed}
          }
      }
    \def\@endtheorem{\global\let\qedsymbol\simpleqedsymbol
      \popQED\endtrivlist\@endpefalse
      \global\let\qedsymbol\qedsymbolOriginal}
  }
  {
    \bool_if:NTF \l__minimalist_use_theorem_in_new_line_bool
      {
        \newtheoremstyle{simple}
          {}{}
          {\normalfont}{}
          {\normalfont}{}
          {\newline}
          {{\thmname{#1}\nobreakspace\thmnumber{#2}}
            {\color{main-text!50!paper}\thmnote{\hspace{.4em}\ensuremath{(\text{#3})}}}\smallskip}
      }
      {
        \newtheoremstyle{simple}
          {}{}
          {\normalfont}{}
          {\normalfont}{}
          {0pt}
          {{\thmname{#1}\nobreakspace\thmnumber{#2}}
            {\color{main-text!50!paper}\thmnote{\hspace{.4em}\ensuremath{(\text{#3})}}}\nobreakspace\nobreakspace{\normalfont\textcolor{main-text!27!paper}{---}}\nobreakspace\nobreakspace}
      }
  }

\theoremstyle{simple}

\renewcommand{\qedsymbol}{
  \makebox[1em]{\color{main-text!27!paper}\rule[-0.1em]{.95em}{.95em}}}
\let\qedsymbolOriginal\qedsymbol

\bool_if:NTF \l__minimalist_fast_bool
  {
    \providecommand{\phantomsection}{}
    \RequirePackage { url }
    \newcommand{\href}[2]{#2}
  }
  {
    \PassOptionsToPackage { hidelinks,linktoc=all } { hyperref }
    \RequirePackage { bookmark }
    \RequirePackage { hyperref }
  }

\PassOptionsToPackage { simplename } { projlib-theorem }
\RequirePackage { projlib-theorem }

\exp_args:No \SetTheorem { \c_projlib_theorem_supported_clist }
  {
    name style = {
      heading style = {
        , english = \minimalist_bfseries:\g_minimalist_title_font_common_tl\textsc
        , french = \minimalist_bfseries:\g_minimalist_title_font_common_tl\textsc
        , ngerman = \minimalist_bfseries:\g_minimalist_title_font_common_tl\textsc
        , italian = \minimalist_bfseries:\g_minimalist_title_font_common_tl\textsc
        , portuguese = \minimalist_bfseries:\g_minimalist_title_font_common_tl\textsc
        , brazilian = \minimalist_bfseries:\g_minimalist_title_font_common_tl\textsc
        , spanish = \minimalist_bfseries:\g_minimalist_title_font_common_tl\textsc
        , schinese = \minimalist_bfseries:\g_minimalist_title_font_common_tl
        , tchinese = \minimalist_bfseries:\g_minimalist_title_font_common_tl
        , japanese = \minimalist_bfseries:\g_minimalist_title_font_common_tl
        , russian = \minimalist_bfseries:\g_minimalist_title_font_common_tl
      }
    }
  }

\SetTheorem { remark }
  {
    name style = {
      heading style = {
        , english = \minimalist_bfseries:\g_minimalist_title_font_common_tl\itshape
        , french = \minimalist_bfseries:\g_minimalist_title_font_common_tl\itshape
        , ngerman = \minimalist_bfseries:\g_minimalist_title_font_common_tl\itshape
        , italian = \minimalist_bfseries:\g_minimalist_title_font_common_tl\itshape
        , portuguese = \minimalist_bfseries:\g_minimalist_title_font_common_tl\itshape
        , brazilian = \minimalist_bfseries:\g_minimalist_title_font_common_tl\itshape
        , spanish = \minimalist_bfseries:\g_minimalist_title_font_common_tl\itshape
        , russian = \minimalist_bfseries:\g_minimalist_title_font_common_tl\itshape
        , schinese = \minimalist_bfseries:\g_minimalist_title_font_common_tl
        , tchinese = \minimalist_bfseries:\g_minimalist_title_font_common_tl
        , japanese = \minimalist_bfseries:\g_minimalist_title_font_common_tl
        , russian = \minimalist_bfseries:\g_minimalist_title_font_common_tl
      }
    }
  }

%%================================
%%  Index
%%================================
\hook_gput_code:nnn { begindocument/before } { minimalist }
{
  \hook_gput_code:nnn { cmd/printindex/before } { minimalist } { \LocallyStopLineNumbers }
  \hook_gput_code:nnn { cmd/printindex/after } { minimalist } { \ResumeLineNumbers }
}


\bool_if:NF \l__minimalist_is_book_bool {

%%================================
%%  Title block style
%%================================
\bool_if:NTF \l__minimalist_classical_bool
  {
    \renewcommand{\@maketitle}{
      \LocallyStopLineNumbers
      \noindent
      {\textcolor{main-text!27!paper}{\rule{\textwidth}{0.75pt}}}
      % \vspace{-\parskip}
      \vspace{-.5\baselineskip}
      \begin{flushright}
        \let\footnote\thanks
        {\minimalist_bfseries:\@title}\\\medskip
        \color{main-text!80!paper}
        {\small\scshape\@author}
        \par\vspace{-\parskip}\vspace{2pt}
        {\small\@date}
      \end{flushright}
      % \vspace{-\parskip}
      \vspace{-.5\baselineskip}
      \ifx\@date\@empty
          \vspace{\baselineskip}
          \vspace{1.2\parskip}
      \else
          \vspace{-.5\baselineskip}
      \fi
      {\textcolor{main-text!27!paper}{\rule{\textwidth}{0.75pt}}\par}
      \ResumeLineNumbers
    }
  }
  {
    \renewcommand{\@maketitle}{
      \LocallyStopLineNumbers
      \begin{center}
        \let\footnote\thanks
        {\minimalist_bfseries:\sffamily\scshape\Large\@title}\\\bigskip
        \color{main-text!80!paper}
        {\small\scshape\@author}
        \par\smallskip\vspace{-\parskip}
        {\small\@date}
      \end{center}
      \ifx\@date\@empty\bigskip\fi
      \bigskip\par
      \ResumeLineNumbers
    }
  }

\hook_gput_code:nnn { cmd/maketitle/after } { minimalist } { \thispagestyle{fancy} }

%%================================
%%  Abstract style
%%================================
\bool_if:NTF \l__minimalist_classical_bool
  {
    \renewenvironment{abstract}
      {
        \LocallyStopLineNumbers
        \begin{flushright}
          \textsc{\minimalist_bfseries:\small\abstractname}\par
          \vspace{-\parskip}
          \vspace{-.25\baselineskip}
          \begin{minipage}[t]{.833\textwidth}
            \vspace{0pt}
            \color{main-text!80!paper}
            \footnotesize
            \parindent=2em
      }
      {
          \end{minipage}
        \end{flushright}
        \bigskip
        \ResumeLineNumbers
      }
  }
  {
    \renewenvironment{abstract}
      {
        \LocallyStopLineNumbers
        \vspace{-\baselineskip}
        \begin{center}
          \textsc{\minimalist_bfseries:\g_minimalist_title_font_common_tl \small\abstractname}\\
          \vspace{-.3\baselineskip}
          \begin{minipage}[t]{.833\textwidth}
            \vspace{0pt}
            \color{main-text!80!paper}
            \footnotesize
            \parindent=2em
      }
      {
          \end{minipage}
        \end{center}
        \medskip
        \ResumeLineNumbers
      }
  }

%%================================
%%  Keyword environment
%%================================
\DefineMultilingualText { \keywordname }
{
  EN = \textsc{Keywords}                      ,
  FR = \textsc{Mots~clés}                     ,
  DE = \textsc{Schlüsselwörter}               ,
  IT = \textsc{Parole~chiave}                 ,
  PT = \textsc{Palavras~chave}                ,
  BR = \textsc{Palavras~chave}                ,
  ES = \textsc{Palabras~clave}                ,
  CN = 关键词                                 ,
  TC = 關鍵詞                                 ,
  JP = キーワード                             ,
  RU = Ключевые~слова                         ,
}

\bool_if:NTF \l__minimalist_classical_bool
  {
    \newenvironment{keyword}{
      \LocallyStopLineNumbers
      \vspace{-.5\baselineskip}
      \begin{flushright}
        {\minimalist_bfseries:\small\keywordname}\par
        \vspace{-\parskip}
        \vspace{-.30\baselineskip}
        \begin{minipage}[t]{.833\textwidth}
          \vspace{0pt}
          \color{main-text!80!paper}
          \footnotesize
          \parindent=2em
          \raggedleft
    }{
        \end{minipage}
      \end{flushright}
      \bigskip
      \ResumeLineNumbers
    }
  }
  {
    \newenvironment{keyword}{
      \LocallyStopLineNumbers
      \vspace{-.75\baselineskip}
      \begin{center}
        {\minimalist_bfseries:\small\keywordname}\\
        \vspace{-.3\baselineskip}
        \begin{minipage}[t]{.833\textwidth}
          \vspace{0pt}
          \color{main-text!80!paper}
          \footnotesize
          \parindent=2em
          \begin{center}
    }{
          \end{center}
        \end{minipage}
      \end{center}
      \medskip
      \ResumeLineNumbers
    }
  }

%%================================
%%  Simulate features of amsart
%%================================
\PassOptionsToPackage { amsfashion } { projlib-author }
\RequirePackage { projlib-author }

}
%</minimalist>

\endinput