% \iffalse meta-comment
%
% Copyright (C) 2021 by Jinwen XU 
% -------------------------------
% 
% This file may be distributed and/or modified under the conditions of the LaTeX
% Project Public License, either version 1.3c of this license or (at your option)
% any later version. The latest version of this license is in:
%
%    http://www.latex-project.org/lppl.txt
%
% \fi
%
%<*driver>
\ProvidesFile{minimalist.dtx}
%</driver>
\NeedsTeXFormat{LaTeX2e}[2020-10-01]
%<*einfart>
\ProvidesClass{einfart}
    [2021/03/12 A simple and clear article style]
%</einfart>
%<*einfartfast>
\ProvidesClass{einfartfast}
    [2021/03/12 A faster but rougher version of einfart]
%</einfartfast>
%<*simplivre>
\ProvidesClass{simplivre}
    [2021/03/12 A simple and clear book style]
%</simplivre>
%<*simplivrefast>
\ProvidesClass{simplivrefast}
    [2021/03/12 A faster but rougher version of simplivre]
%</simplivrefast>

%<*einfart|einfartfast>
\DeclareOption*{%
    \PassOptionsToClass{\CurrentOption}{article}}
\ProcessOptions\relax
\LoadClass{article}
%</einfart|einfartfast>
%<*simplivre|simplivrefast>
\DeclareOption*{%
    \PassOptionsToClass{\CurrentOption}{book}}
\ProcessOptions\relax
\LoadClass{book}
%</simplivre|simplivrefast>

\RequirePackage{etoolbox}

%%================================
%% Fonts
%%================================
\RequirePackage{anyfontsize}

%% Title fonts
\newcommand{\partfont}{\sffamily}
%<simplivre|simplivrefast>\newcommand{\chapfont}{\sffamily}
\newcommand{\secfont}{}
\newcommand{\subsecfont}{}
\newcommand{\subsubsecfont}{}
%<*einfartfast|simplivrefast>

%% Math fonts
\RequirePackage{mathpazo}
%</einfartfast|simplivrefast>

%% English fonts
\PassOptionsToPackage{no-math}{fontspec}
\RequirePackage{fontspec}
\IfFontExistsTF{Palatino Linotype}{%
    \setmainfont{Palatino Linotype}
}{
    \setmainfont{texgyrepagella-regular.otf}[
        BoldFont       = texgyrepagella-bold.otf ,
        ItalicFont     = texgyrepagella-italic.otf ,
        BoldItalicFont = texgyrepagella-bolditalic.otf ]
}
    \setsansfont{SourceSansPro-Regular.otf}[
        Scale          = MatchLowercase,
        BoldFont       = SourceSansPro-Bold.otf ,
        ItalicFont     = SourceSansPro-RegularIt.otf ,
        BoldItalicFont = SourceSansPro-BoldIt.otf ]

%% Chinese fonts
\PassOptionsToPackage{fontset=none,scheme=plain}{ctex}
\RequirePackage{ctex}
\IfFontExistsTF{FZYouSongS 507R}{%
    \setCJKmainfont{FZYouSongS 507R}[
        BoldFont       = FZYouSongS 509R ,
        BoldFeatures   = {FakeBold=2} ,
        ItalicFont     = * ,
        BoldItalicFont = FZYouSongS 509R ,
        BoldItalicFeatures = {FakeBold=2} ,
        SmallCapsFont  = * ]
}{
    \setCJKmainfont{FandolSong-Regular.otf}[
        BoldFont       = FandolSong-Bold.otf ,
        ItalicFont     = FandolKai-Regular.otf ,
        BoldItalicFont = FandolKai-Regular.otf ,
        BoldItalicFeatures = {FakeBold=4} ,
        SmallCapsFont  = * ]
}
\IfFontExistsTF{FZYouSongS 507R}{%
    \setCJKmonofont{FZYouSongS 507R}[
        BoldFont       = FZYouSongS 509R ,
        BoldFeatures   = {FakeBold=2} ,
        ItalicFont     = * ,
        BoldItalicFont = FZYouSongS 509R ,
        BoldItalicFeatures = {FakeBold=2} ,
        SmallCapsFont  = * ]
}{
    \setCJKmonofont{FandolFang-Regular.otf}[
        BoldFont       = * ,
        BoldFeatures   = {FakeBold=4} ,
        ItalicFont     = * ,
        BoldItalicFont = * ,
        BoldItalicFeatures = {FakeBold=4} ,
        SmallCapsFont  = * ]
}
\IfFontExistsTF{FZYouHeiS 506L}{%
    \setCJKsansfont{FZYouHeiS 506L}[
        BoldFont       = FZYouHeiS 509R,
        ItalicFont     = * ,
        BoldItalicFont = FZYouHeiS 509R ,
        SmallCapsFont  = * ]
}{
    \setCJKsansfont{FandolHei-Regular.otf}[
        BoldFont       = FandolHei-Bold.otf ,
        ItalicFont     = * ,
        BoldItalicFont = FandolHei-Bold.otf ,
        SmallCapsFont  = * ]
}
%<*einfart|simplivre>

%% Math fonts
\PassOptionsToPackage
    {warnings-off={mathtools-colon,mathtools-overbracket}}{unicode-math}
\RequirePackage{unicode-math}
\unimathsetup{math-style=ISO}
\setmathfont{Asana-Math.otf}
\IfFontExistsTF{Neo Euler}{%
\setmathfont{Neo Euler} % From https://tex.stackexchange.com/a/425887
    [range={"0000-"0001,"0020-"007E,
            "00A0,"00A7-"00A8,"00AC,"00AF,"00B1,"00B4-"00B5,"00B7,
            "00D7,"00F7,
            "0131,
            "0237,"02C6-"02C7,"02D8-"02DA,"02DC,
            "0300-"030C,"030F,"0311,"0323-"0325,"032E-"0332,"0338,
            "0391-"0393,"0395-"03A1,"03A3-"03A8,"03B1-"03BB,
            "03BD-"03C1,"03C3-"03C9,"03D1,"03D5-"03D6,"03F5,
            "2016,"2018-"2019,"2021,"2026-"202C,"2032-"2037,"2044,
            "2057,"20D6-"20D7,"20DB-"20DD,"20E1,"20EE-"20EF,
            "210B-"210C,"210E-"2113,"2118,"211B-"211C,"2126-"2128,
            "212C-"212D,"2130-"2131,"2133,"2135,"2190-"2199,
            "21A4,"21A6,"21A9-"21AA,"21BC-"21CC,"21D0-"21D5,
            "2200,"2202-"2209,"220B-"220C,"220F-"2213,"2215-"221E,
            "2223,"2225,"2227-"222E,"2234-"2235,"2237-"223D,
            "2240-"224C,"2260-"2269,"226E-"2279,"2282-"228B,"228E,
            "2291-"2292,"2295-"2299,"22A2-"22A5,"22C0-"22C5,
            "22DC-"22DD,"22EF,"22F0-"22F1,
            "2308-"230B,"2320-"2321,"2329-"232A,"239B-"23AE,
            "23DC-"23DF,
            "27E8-"27E9,"27F5-"27FE,"2A0C,"2B1A,
            "1D400-"1D433,"1D49C,"1D49E-"1D49F,"1D4A2,"1D4A5-"1D4A6,
            "1D4A9-"1D4AC,"1D4AE-"1D4B5,"1D4D0-"1D4E9,"1D504-"1D505,
            "1D507-"1D50A,"1D50D-"1D514,"1D516-"1D51C,"1D51E-"1D537,
            "1D56C-"1D59F,"1D6A8-"1D6B8,"1D6BA-"1D6D2,"1D6D4-"1D6DD,
            "1D6DF,"1D6E1,"1D7CE-"1D7D7 }]
}{}
%</einfart|simplivre>

\RequirePackage[verbose=silent]{microtype}

%%================================
%% Page layout
%%================================
\RequirePackage[heightrounded]{geometry}
\geometry{
    papersize={7in,10in},
    total={40em,60em},
    hmarginratio=1:1,
    vmarginratio=1:1,
    footnotesep=2em plus 2pt minus 2pt,
}

\RequirePackage{xcolor}
\definecolor{paper}{RGB}{255,255,255}
%<*simplivre>

\RequirePackage{tikz}
\usetikzlibrary{calc,shadings}
\RequirePackage{tikzpagenodes}% For `current page text area`
\newcommand{\drawHelpLine}{%
    \begin{tikzpicture}[remember picture,overlay]
        \foreach\i in {0,1,...,5}{%
            \fill[opacity=0.12-0.02*\i] 
                ($(current page text area.north east)
                    +(-\i*0.5em-.025em,-10pt+\i*1.1pt)$) 
                rectangle ($(current page text area.south east)
                    +(-\i*0.5em+.025em,10pt-\i*1.1pt)$);
            \shade[top color=paper,bottom color=black,opacity=0.12-0.02*\i] 
                ($(current page text area.north east)
                    +(-\i*0.5em-.025em,2pt)$) 
                rectangle ($(current page text area.north east)
                    +(-\i*0.5em+.025em,-10pt+\i*1.1pt)$);
            \shade[top color=black,bottom color=paper,opacity=0.12-0.02*\i] 
                ($(current page text area.south east)
                    +(-\i*0.5em-.025em,-2pt)$) 
                rectangle ($(current page text area.south east)
                    +(-\i*0.5em+.025em,10pt-\i*1.1pt)$);
        }
    \end{tikzpicture}%
}
%</simplivre>
%<*simplivrefast>

\newcommand{\drawHelpLine}{}
%</simplivrefast>

\RequirePackage{fancyhdr}
\RequirePackage{extramarks}
\fancypagestyle{fancy}{
    \fancyhf{}
    \if@twoside
        \fancyfoot[RO]{\small\textcolor{black!30!paper}{\lastrightmark}%
            ~~\rlap{\textcolor{gray!55!paper}{$|$}~~\thepage}}
        \fancyfoot[LE]{\small\leavevmode\llap{\thepage%
            ~~\textcolor{gray!55!paper}{$|$}}%
            ~~\textcolor{black!30!paper}{\lastleftmark}}
    \else
        \fancyfoot[R]{\small\textcolor{black!30!paper}{\lastrightmark}%
            ~~\rlap{\textcolor{gray!55!paper}{$|$}~~\thepage}}
    \fi
    \renewcommand{\headrulewidth}{0pt}
}
\pagestyle{fancy}

\fancypagestyle{plain}{
    \fancyhf{}
    \if@twoside
        \fancyfoot[RO]{\small%
            ~\rlap{\textcolor{gray!55!paper}{$|$}~~\thepage}}
        \fancyfoot[LE]{\small\leavevmode\llap{\thepage%
            ~~\textcolor{gray!55!paper}{$|$}}}
    \else
        \fancyfoot[R]{\small%
            ~\rlap{\textcolor{gray!55!paper}{$|$}~~\thepage}}
    \fi
    \renewcommand{\headrulewidth}{0pt}
}

%<*einfart|einfartfast>
\if@twoside
    \renewcommand*{\sectionmark}[1]{\markboth{\textsc{#1}}{}}
\else
    \renewcommand*{\sectionmark}[1]{\markboth{\textsc{#1}}{\textsc{#1}}}
\fi
%</einfart|einfartfast>
%<*simplivre|simplivrefast>
\if@twoside
    \renewcommand{\chaptermark}[1]{\markboth{\textsc{#1}}{}}
\else
    \renewcommand{\chaptermark}[1]{\markboth{\textsc{#1}}{\textsc{#1}}}
\fi
\renewcommand*{\sectionmark}[1]{%
    \markright{\raisebox{.03em}{\footnotesize/}%
    ~\thesection~\raisebox{.03em}{\footnotesize/}~~~#1}}

\fancypagestyle{part}{
    \fancyhf{}
    \renewcommand{\headrulewidth}{0pt}
    \fancyhead[C]{\drawHelpLine}
}

\addtolength{\headheight}{20pt}
\addtolength{\topmargin}{-20pt}
%</simplivre|simplivrefast>

%%================================
%% Line spacing
%%================================
\RequirePackage{setspace}
\setstretch{1.07}
% To avoid `Underfull \vbox (badness 10000)`
\raggedbottom

\AtEndPreamble{\RequirePackage{parskip}}

%%================================
%% Line numbers
%%================================
\PassOptionsToPackage{pagewise,mathlines}{lineno}
\RequirePackage{lineno}
\renewcommand\linenumberfont{\ttfamily\color{gray!15!paper}\footnotesize}
\setlength\linenumbersep{1em}

\RequirePackage{mathtools}

%% From https://tex.stackexchange.com/a/461192
% Patch 'normal' math environments:
\newcommand*\linenomathpatch[1]{%
    \cspreto{#1}{\linenomath}%
    \cspreto{#1*}{\linenomath}%
    \cspreto{end#1}{\endlinenomath}%
    \cspreto{end#1*}{\endlinenomath}%
}
% Patch AMS math environments:
\newcommand*\linenomathpatchAMS[1]{%
    \cspreto{#1}{\linenomathAMS}%
    \cspreto{#1*}{\linenomathAMS}%
    \csappto{end#1}{\endlinenomath}%
    \csappto{end#1*}{\endlinenomath}%
}
% Define \linenomathAMS depending on whether 'mathlines' option is provided
\expandafter\ifx\linenomath\linenomathWithnumbers
    \let\linenomathAMS\linenomathWithnumbers
% The following line gets rid of an extra line numbers at the bottom:
    \patchcmd\linenomathAMS{\advance\postdisplaypenalty\linenopenalty}{}{}{}
\else
    \let\linenomathAMS\linenomathNonumbers
\fi

\linenomathpatch{equation}
\linenomathpatchAMS{gather}
\linenomathpatchAMS{multline}
\linenomathpatchAMS{align}
\linenomathpatchAMS{alignat}
\linenomathpatchAMS{flalign}

% record whether linenumber has turned on
\newif\ifLNturnsON
\def\LocallyStopLineNumbers{\LNturnsONfalse%
    \ifLineNumbers\LNturnsONtrue\fi\nolinenumbers}
\def\ResumeLineNumbers{\ifLNturnsON\linenumbers\fi}
% switch off the line numbers of TOC
\pretocmd{\tableofcontents}{\LocallyStopLineNumbers}{}{\FAIL}
\apptocmd{\tableofcontents}{\ResumeLineNumbers}{}{\FAIL}
% switch off the line numbers of bibliography
\pretocmd{\thebibliography}{\LocallyStopLineNumbers}{}{\FAIL}
\apptocmd{\endthebibliography}{\ResumeLineNumbers}{}{\FAIL}


%%================================
%% Title format
%%================================
\RequirePackage[explicit,newparttoc]{titlesec}
\PassOptionsToPackage{normalem}{ulem}
\RequirePackage{ulem}

%<*einfart|einfartfast>
%% Part
\titleformat{\part}[display]
    {\LocallyStopLineNumbers%
    \partfont\filleft}
    {\MakeUppercase{\partname~\protect\thepart}}
    {.3em}
    {\fontsize{16}{0}\selectfont\MakeUppercase{#1}}
    [\ResumeLineNumbers]
\titleformat{name=\part,numberless}[display]
    {\LocallyStopLineNumbers%
    % \phantomsection\addcontentsline{toc}{part}{#1}%
    \partfont\filleft}
    {\phantom{\MakeUppercase{\partname}}}
    {.3em}
    {\fontsize{16}{0}\selectfont\MakeUppercase{#1}}
    [\ResumeLineNumbers]
%% Text after part
\newcommand{\parttext}[1]{%
    \LocallyStopLineNumbers%
    \begin{flushright}%
        \begin{minipage}{0.833\textwidth}%
            \color{black!80!paper}\raggedleft#1%
        \end{minipage}%
    \end{flushright}%
    \ResumeLineNumbers%
}
%</einfart|einfartfast>
%<*simplivre|simplivrefast>
%% Part
\titleclass{\part}{top} % make part like a chapter
\titleformat{\part}[display]
    {\thispagestyle{part}%
    \LocallyStopLineNumbers%
    \partfont\filleft}
    {\MakeUppercase{\partname~\protect\thepart}}
    {1em}
    {\fontsize{20}{0}\selectfont\MakeUppercase{#1}}
    [\ResumeLineNumbers]
\titleformat{name=\part,numberless}[display]
    {\thispagestyle{part}%
    \LocallyStopLineNumbers%
    % \phantomsection\addcontentsline{toc}{part}{#1}%
    \partfont\filleft}
    {\phantom{\MakeUppercase{\partname}}}
    {1em}
    {\fontsize{20}{0}\selectfont\MakeUppercase{#1}}
    [\ResumeLineNumbers]
\titlespacing*{\part}{0pt}{5em}{6em}
%% Text after part
\newcommand{\parttext}[1]{%
\vfill%
\LocallyStopLineNumbers%
\begin{flushright}%
    \begin{minipage}{0.833\textwidth}%
        \color{black!80!paper}\raggedleft#1%
    \end{minipage}%
\end{flushright}%
\ResumeLineNumbers%
\vfill\vfill%
\cleardoublepage%
}

%% Chapter
\titleformat{\chapter}
    {\thispagestyle{fancy}%
    \LocallyStopLineNumbers%
    \color{black!80!paper}\chapfont\fontsize{16}{0}\selectfont}{}{0em}
    {\rlap{\hspace*{-.5em}{\color{gray!25!paper}%
        \fontsize{80}{0}\selectfont\raisebox{-7pt}{\thechapter}}}#1}
    [\ResumeLineNumbers]
\titleformat{name=\chapter,numberless}
    {\thispagestyle{fancy}%
    \LocallyStopLineNumbers%
    % \phantomsection\addcontentsline{toc}{chapter}{#1}%
    \color{black!80!paper}\chapfont\fontsize{16}{0}\selectfont}{}{0em}
    {\rlap{\hspace*{-.5em}{\color{gray!25!paper}%
        \fontsize{80}{0}\selectfont\normalfont\raisebox{-7pt}{*}}}#1}
    [\ResumeLineNumbers]
%</simplivre|simplivrefast>

%% Section
\newcommand\seculine{\bgroup\markoverwith{\color{gray!55!paper}%
    \rule[-0.9ex]{2pt}{.6pt}\hspace{-2pt}\rule[-1.2ex]{2pt}{.6pt}}\ULon}
\renewcommand\thesection{\arabic{section}}
\titleformat{\section}
    {\LocallyStopLineNumbers%
    \secfont\centering}{}{0em}
    {{\small\textcolor{gray!55!paper}{\raisebox{.03em}{\footnotesize/}}%
        \,\,\textcolor{black!80!paper}{\arabic{section}}%
        \,\,\textcolor{gray!55!paper}{\raisebox{.03em}{\footnotesize/}}}\\
        \seculine{#1}}
    [\ResumeLineNumbers]
\titleformat{name=\section,numberless}
    {\LocallyStopLineNumbers%
    % \phantomsection\addcontentsline{toc}{section}{#1}%
    \secfont\centering}{}{0em}
    {\seculine{#1}}
    [\ResumeLineNumbers]

%% Subsection
\newcommand\subseculine{\bgroup\markoverwith{\color{gray!55!paper}%
    \rule[-1ex]{2pt}{.75pt}}\ULon}
\renewcommand\thesubsection{%
    \ifnum\c@section=0\else\arabic{section}.\fi\arabic{subsection}}
\titleformat{\subsection}
    {\LocallyStopLineNumbers%
    \subsecfont}{}{0em}
    {\subseculine{\thesubsection~\textcolor{gray!55!paper}{$|$}~#1}}
    [\ResumeLineNumbers]
\titleformat{name=\subsection,numberless}
    {\LocallyStopLineNumbers%
    \subsecfont}{}{0em}
    {\subseculine{#1}}
    [\ResumeLineNumbers]

%% Subsubsection
\titleformat{\subsubsection}
    {\LocallyStopLineNumbers%
    \subsubsecfont}{\thesubsubsection}{1em}
    {#1}
    [\ResumeLineNumbers]
\titlespacing{\subsubsection}{0pt}{.8\baselineskip}{.5\baselineskip}

%%================================
%% TOC format
%%================================
\RequirePackage{titletoc}
\titlecontents{part}
    [0em]
    {\addvspace{1.5pc}\filcenter\partfont}
    {\thecontentslabel\\\uppercase}
    {}
    {} % without page number
    [\addvspace{.5pc}]
%<*einfart|einfartfast>
\titlecontents{section}
    [2em] % i.e., 0em (part) + 2em
    {\secfont}
    {\contentslabel{1.75em}}
    {\hspace*{-1.75em}}
    {\titlerule*[1em]{\textcolor{gray!30!paper}{.}}\contentspage}
\titlecontents{subsection}
    [5em] % i.e., 2em (section) + 3em
    {\subsecfont}
    {\contentslabel{2.75em}}
    {\hspace*{-2.75em}}
    {\titlerule*[1em]{\textcolor{gray!30!paper}{.}}\contentspage}
%</einfart|einfartfast>
%<*simplivre|simplivrefast>
\titlecontents{chapter}
    [2em] % i.e., 0em (part) + 2em
    {\addvspace{.5pc}\chapfont}
    {\contentslabel{2em}}
    {\hspace*{-2em}}
    {\normalfont\titlerule*[1em]{\textcolor{gray!30!paper}{.}}\contentspage}
\titlecontents{section}
    [4em] % i.e., 2em (chapter) + 2em
    {\secfont}
    {\contentslabel{1.75em}}
    {\hspace*{-1.75em}}
    {\titlerule*[1em]{\textcolor{gray!30!paper}{.}}\contentspage}
\titlecontents{subsection}
    [7em] % i.e., 4em (section) + 3em
    {\subsecfont}
    {\contentslabel{2.75em}}
    {\hspace*{-2.75em}}
    {\titlerule*[1em]{\textcolor{gray!30!paper}{.}}\contentspage}
%</simplivre|simplivrefast>

%%================================
%% Graphics
%%================================
\RequirePackage{graphicx}
\graphicspath{{images/}}
\RequirePackage{wrapfig}
\RequirePackage{caption}

%%================================
%% Lists
%%================================
\RequirePackage{enumitem}
\setlist{noitemsep,leftmargin=2em}
\renewcommand\labelitemi{\color{gray!50}$\bullet$} 

%%================================
%% Blank page
%%================================
\newcommand{\blinkpagetext}{This page is intentionally left blank}
\renewcommand{\cleardoublepage}{\relax
    \clearpage
    \if@twoside\ifodd\c@page\relax\else
    \thispagestyle{empty}
    \newgeometry{centering}
    \null\vfill
    \centerline{\large\color{gray!20!paper}\blinkpagetext}
    \vfill\restoregeometry\newpage\fi\fi}

%%================================
%% Index
%%================================
\RequirePackage{imakeidx}
% switch off the line numbers of index
\pretocmd{\printindex}{\LocallyStopLineNumbers}{}{\FAIL}
\apptocmd{\printindex}{\ResumeLineNumbers}{}{\FAIL}

%%================================
%% Theorems
%%================================
\RequirePackage{amsthm}
\newtheoremstyle{simple}%
    {}{}%
    {\normalfont}{}%
    {\normalfont}{}%
    {0pt}%
    {\thmname{\textsc{#1}}\thmnumber{ #2}\hspace{.4em}%
        \textcolor{gray!55!paper}{$|$}\hspace{.4em}%
        \color{gray}\thmnote{\ensuremath{(\text{#3})}~~}\pushQED{\qed}}
\def\@endtheorem{\popQED\endtrivlist\@endpefalse }

\renewcommand{\qedsymbol}{%
    \makebox[1em]{\color{gray!55!paper}\rule[-0.1em]{.95em}{.95em}}}

%<einfart|simplivre>\PassOptionsToPackage{hidelinks,linktoc=all}{hyperref}
\RequirePackage{aliascnt}
% To solve `Difference between bookmark levels is greater than one`
%<einfart|simplivre>\RequirePackage{bookmark}
%<einfart|simplivre>\RequirePackage{hyperref}
\PassOptionsToPackage{nameinlink}{cleveref}
\RequirePackage{cleveref}

\newcommand\englishABBR{EN}
\newcommand\frenchABBR{FR}
\newcommand\chineseABBR{CN}

%% Macro for creating theorems
\RequirePackage{xstring}
\newcommand\PassFirstToSecond[2]{#2{#1}}%
\NewDocumentCommand{\CreateTheorem}{sm}{%
    \begingroup
    \protected@edef\temp{#2}%
    \expandafter\IfEndWith\expandafter{\temp}{*}{%
        \expandafter\StrGobbleRight\expandafter{\temp}{1}[\temp]%
        \PassFirstToSecond{*}%
    }{%
        \PassFirstToSecond{}%
    }%
    {\expandafter\PassFirstToSecond%
        \expandafter{\temp}{\endgroup\InnerCreateTheorem{#1}}}%
}%
\NewDocumentCommand{\InnerCreateTheorem}{mmmod<>}{%
    % #1 = star or no star
    % #2 = name of environment
    % #3 = emptiness or star to append to name of environment
    % #4 = numbered like
    % #5 = numbered within
    \IfBooleanTF{#1}{%
        \IfValueTF{#4}
            {\@firstoftwo}
            {\IfValueTF{#5}{\@firstoftwo}{\@secondoftwo}}%
    }{%
        \IfValueTF{#4}
            {\IfValueTF{#5}{\@firstoftwo}{\@secondoftwo}}{
            \@secondoftwo}
    }%
    {%
        \GenericError{}%
        {\string\CreateTheorem\space syntax error\on@line}{%
        You cannot call the starred variant with optional argument,\MessageBreak
        nor call the unstarred variant with several optional arguments.}%
        {}%
    }{%
        \IfBooleanTF{#1}{%
            \newtheorem*{#2EN#3}{\csname#2nameEN\endcsname}
            \newtheorem*{#2FR#3}{\csname#2nameFR\endcsname}
            \newtheorem*{#2CN#3}{\csname#2nameCN\endcsname}
        }{%
            \IfValueTF{#5}{%
                \newcounter{#2#3}[{#5}]%
                \expandafter\renewcommand\expandafter*%
                    \csname the#2#3\expandafter\endcsname%
                    \expandafter{\csname the#5\endcsname.\arabic{#2#3}}%
            }{%
                \IfValueTF{#4}
                    {\newaliascnt{#2#3}{#4}}
                    {\newcounter{#2#3}}%
            }%
            %-------------------------------------------------------------------
            \CreateTheoremNumberedLikeAliasCounter{#2}{EN}{#3}%
            \CreateTheoremNumberedLikeAliasCounter{#2}{FR}{#3}%
            \CreateTheoremNumberedLikeAliasCounter{#2}{CN}{#3}%
            %-------------------------------------------------------------------
        }%
        \NewDocumentEnvironment{#2#3}{}
            {\csname#2\csname\languagename ABBR\endcsname#3\endcsname}%
            {\csname end#2\csname\languagename ABBR\endcsname#3\endcsname}%
    }%
}%
\NewDocumentCommand{\CreateTheoremNumberedLikeAliasCounter}{mmm}{%
    \newaliascnt{#1#2#3}{#1#3}%
    \newtheorem{#1#2#3}[{#1#2#3}]{\csname#1name#2\endcsname}%
    \aliascntresetthe{#1#2#3}%
    \crefname{#1#2#3}%
        {\csname#1name#2\endcsname}%
        {\csname#1name#2\endcsname}%
}%

%% English theorems names
\def\theoremnameEN{\textsc{Theorem}}
\def\lemmanameEN{\textsc{Lemma}}
\def\propositionnameEN{\textsc{Proposition}}
\def\corollarynameEN{\textsc{Corollary}}
\def\factnameEN{\textsc{Fact}}
\def\conjecturenameEN{\textsc{Conjecture}}
\def\definitionnameEN{\textsc{Definition}}
\def\examplenameEN{\textsc{Example}}
\def\problemnameEN{\textsc{Problem}}
\def\remarknameEN{\textsc{Remark}}

%% French theorems names
\def\theoremnameFR{\textsc{Théorème}}
\def\lemmanameFR{\textsc{Lemme}}
\def\propositionnameFR{\textsc{Proposition}}
\def\corollarynameFR{\textsc{Corollaire}}
\def\factnameFR{\textsc{Fait}}
\def\conjecturenameFR{\textsc{Conjecture}}
\def\definitionnameFR{\textsc{Définition}}
\def\examplenameFR{\textsc{Exemple}}
\def\problemnameFR{\textsc{Problème}}
\def\remarknameFR{\textsc{Remarque}}

%% Chinese theorems names
\def\theoremnameCN{定理}
\def\lemmanameCN{引理}
\def\propositionnameCN{命题}
\def\corollarynameCN{推论}
\def\factnameCN{事实}
\def\conjecturenameCN{猜想}
\def\definitionnameCN{定义}
\def\examplenameCN{例}
\def\problemnameCN{问题}
\def\remarknameCN{备注}

%% Theorem environments
\theoremstyle{simple}
%<einfart|einfartfast>\CreateTheorem{theorem}<section>
%<simplivre|simplivrefast>\CreateTheorem{theorem}<chapter>
\CreateTheorem{lemma}[theorem]
\CreateTheorem{proposition}[theorem]
\CreateTheorem{corollary}[theorem]
\CreateTheorem{fact}[theorem]
%<einfart|einfartfast>\CreateTheorem{conjecture}<section>
%<simplivre|simplivrefast>\CreateTheorem{conjecture}<chapter>
\CreateTheorem*{theorem*}
\CreateTheorem*{lemma*}
\CreateTheorem*{proposition*}
\CreateTheorem*{corollary*}
\CreateTheorem*{fact*}
\CreateTheorem*{conjecture*}
\CreateTheorem{definition}[theorem]
%<einfart|einfartfast>\CreateTheorem{example}<section>
%<simplivre|simplivrefast>\CreateTheorem{example}<chapter>
%<einfart|einfartfast>\CreateTheorem{problem}<section>
%<simplivre|simplivrefast>\CreateTheorem{problem}<chapter>
\CreateTheorem*{definition*}
\CreateTheorem*{example*}
\CreateTheorem*{problem*}
%<einfart|einfartfast>\CreateTheorem{remark}<section>
%<simplivre|simplivrefast>\CreateTheorem{remark}<chapter>
\CreateTheorem*{remark*}

%%================================
%% Language configuration
%%================================
%<*einfart|simplivre>
\PassOptionsToPackage{french,english}{babel}
\RequirePackage{babel}
\frenchsetup{PartNameFull=false}
%</einfart|simplivre>
%<*einfartfast|simplivrefast>
\RequirePackage{polyglossia}
\setdefaultlanguage{english}
\setotherlanguage[frenchpart=false]{french}
%</einfartfast|simplivrefast>

\newcommand{\minimalist@langconfig@chinese}{%
    \def\abstractname{摘要}%
    \def\proofname{证明}%
    \def\contentsname{目录}%
    \def\listfigurename{插图}%
    \def\listtablename{表格}%
    \def\figurename{图}%
    \def\tablename{表}%
    \def\indexname{索引}%
    \def\appendixname{附录}%
    \def\bibname{参考文献}%
    \renewcommand{\languagename}{chinese}%
}
\newcommand{\minimalist@langconfig@english}{%
    \selectlanguage{english}%
}
\newcommand{\minimalist@langconfig@french}{%
    \selectlanguage{french}%
% The line below is only needed for 'babel'
%<einfart|simplivre>    \def\frenchpartname{Partie}%
}

\newcommand{\UseLanguageCORE}[1]{%
    \ifstrequal{#1}{chinese}{\onehalfspacing\minimalist@langconfig@chinese}{}%
    \ifstrequal{#1}{Chinese}{\onehalfspacing\minimalist@langconfig@chinese}{}%
    \ifstrequal{#1}{english}{\setstretch{1.07}\minimalist@langconfig@english}{}%
    \ifstrequal{#1}{English}{\setstretch{1.07}\minimalist@langconfig@english}{}%
    \ifstrequal{#1}{french}{\setstretch{1.07}\minimalist@langconfig@french}{}%
    \ifstrequal{#1}{French}{\setstretch{1.07}\minimalist@langconfig@french}{}%
}
\newcommand{\UseLanguage}[1]{%
    \ifx\@onlypreamble\@notprerr%
        \UseLanguageCORE{#1}%
    \else%
        \AfterEndPreamble{\UseLanguageCORE{#1}}%
    \fi%
}
\newcommand{\UseOtherLanguage}[2]{%
\begingroup%
    \ifstrequal{#1}{chinese}{\minimalist@langconfig@chinese}{}%
    \ifstrequal{#1}{Chinese}{\minimalist@langconfig@chinese}{}%
    \ifstrequal{#1}{english}{\minimalist@langconfig@english}{}%
    \ifstrequal{#1}{English}{\minimalist@langconfig@english}{}%
    \ifstrequal{#1}{french}{\minimalist@langconfig@french}{}%
    \ifstrequal{#1}{French}{\minimalist@langconfig@french}{}%
    #2%
\endgroup%
}

%%================================
%% Draft mark
%%================================
\def\dnfFont{\ttfamily}
\def\needgraphFont{\ttfamily}

\def\dnfTextEN{To be finished here}
\def\needgraphTextEN{A graph is needed here}
\def\dnfTextFR{À terminer ici}
\def\needgraphTextFR{Il manque encore un graphique ici}
\def\dnfTextCN{这里的内容尚未完成}
\def\needgraphTextCN{这里需要一张图片}

\definecolor{dnfColor}{RGB}{21,122,20}
\definecolor{needgraphColor}{RGB}{70,130,180}

%<*einfart|simplivre>
\PassOptionsToPackage{many}{tcolorbox}
\RequirePackage{tcolorbox}
\newtcbox{\plainBox}[1][-paper]{enhanced jigsaw,%
    on line, arc = 1.2pt, outer arc = 1pt,breakable,%
    colframe = #1,colupper=#1,opacityback=0,%
    boxsep = 1pt,boxrule = 1.2pt,%
    left = 1pt, right = 1pt, top = 0pt, bottom = 0pt,%
}
%</einfart|simplivre>
%<*einfartfast|simplivrefast>
\newcommand{\plainBox}[2][-paper]{\textcolor{#1}{%
    \setlength{\fboxsep}{1.5pt}%
    \setlength{\fboxrule}{1.2pt}%
    \fbox{#2}}}
%</einfartfast|simplivrefast>

\NewDocumentCommand{\dnf}{d<>}{%
    \noindent\plainBox[dnfColor]%
    {\normalfont\dnfFont\bfseries\small%
    \csname dnfText\csname\languagename ABBR\endcsname\endcsname%
    \IfNoValueF{#1}{ : #1}}%
}
\NewDocumentCommand{\needgraph}{d<>}{%
    \par%
    \centerline{\plainBox[needgraphColor]%
    {\normalfont\needgraphFont\bfseries\small%
    \csname needgraphText\csname\languagename ABBR\endcsname\endcsname%
    \IfNoValueF{#1}{ : #1}}}%
    \par%
}
%<*einfart|einfartfast>

%%================================
%% Title block style
%%================================
\renewcommand{\@maketitle}{%
    \LocallyStopLineNumbers%
    \noindent%
    {\textcolor{gray!55!paper}{\rule{\textwidth}{0.75pt}}}%
    \vspace{-\parskip}%
    \begin{flushright}%
        {\@title}\\\medskip%
        \color{black!80!paper}%
        {\small\scshape\@author}\\[2pt]%
        {\small\@date}%
    \end{flushright}%
    \vspace{-\parskip}%
    \vspace{-.5\baselineskip}%
    {\textcolor{gray!55!paper}{\rule{\textwidth}{0.75pt}}\par}%
    \ResumeLineNumbers%
}
\apptocmd{\maketitle}{\thispagestyle{fancy}}{}{\FAIL}

%%================================
%% Abstract style
%%================================
\renewenvironment{abstract}{%
    \LocallyStopLineNumbers%
    \begin{flushright}%
        \textsc{\small\abstractname}\par%
        \vspace{-.75\baselineskip}%
        \begin{minipage}[t]{.833\textwidth}%
            \vspace{0pt}%
            \color{black!80!paper}%
            \footnotesize%
            \parindent=2em
}{%
        \end{minipage}%
    \end{flushright}%
    \ResumeLineNumbers%
}
%</einfart|einfartfast>

\endinput